\documentclass[english]{hiscover}

% The file 'metadata.tex' contains the user's metadata
% such as title, name, ISBN, ...
\title{Game-calibrated and user-tailored remote detection of emotions}
\subtitle{A non-intrusive, multifactorial camera-based approach for detecting stress and boredom of players in games}
\subject{Informatics}
\date{2018}{10}{14}
\isbn{978-91-984187-9-8}
\printshop{BrandFactory AB, Gothenburg}
\seriesnumber{27}% in University of Skövde's 'dissertation series'
\author{Fernando Bevilacqua}
\decidedby{den internationella kommittéen för den fjärde internationalen}
\defensedaytimeroom{mån}{19}{november}{2018}{13.00}{Portalen, Insikten}
\opponent{Vittorio Scarano, University of Salerno}
%% \dissertationtype is the official name of the publication's
%% type, either in Swedish or English (whatever applies to you).
%% The value set here is used among others on the title page.
%% Not to be mixed up with \publicationtype
% \dissertationtype{licentiatexamen}
\dissertationtype{filosofie doktorsexamen}
\dissertationarea{informationsteknologi}
\spokenlanguage{engelska}
%\spokenlanguage{svenska}
%% \publicationtype controls whether the document is a
%% dissertation or a licentiate thesis. It has to be one
%% of the two values. It determines various formatting
%% aspects, such as the default colors (purple vs. grey).
%% Not to be mixed up with \dissertationtype
\publicationtype{dissertation}% or \publicationtype{licentiate} or \publicationtype{thesisproposal} or \publicationtype{researchproposal}


\begin{document}
\cover{%
% First argument: your cover image, e.g. \includegraphics{..} or \begin{tikzpicture}...\end{tikzpicture}
% Image should be 13.75cm x 8cm (ratio of 55:32)
\includegraphics[width=13.75cm,height=8cm]{foto}%
% No image? Just use 8cm of vertical space:
%\vspace*{8cm}
}{%
% Second argument: filename of a photo of you
face%
}{%
% Third argument: the abstract to be shown on the back of the book
Fernando Bevilacqua has a background in Computer Science (BSc), and Computation
focused on Computer Graphics (MSc) from the Federal University of Santa Maria (UFSM),
Brazil. In 2003 he co-founded Decadium Studios, a video game
development company, where he worked for years in variety of applications and
games. In 2010 he joined academia by becoming a lecturer of Computer Science at
Federal University of Fronteira Sul (UFFS), in Chapec{\'o}, Brazil. Fernando is
an open-source evangelist whose academic interests include Games Research,
Human-Computer Interaction, and Computer Vision.
\\
\\
In this thesis, Fernando presents a method for remote detection of emotional
states (boredom and stress) of a person during the interaction with
games. Player's heart rate and facial actions are used in the process, which are
automatically and remotely calculated from a video feed using Computer Vision.
Since the information is acquired remotely, no specialized equipment like physical
sensors is needed, only an ordinary camera and a computer are required.
Fernando supports the creation and testing of the proposed method with a
series of systematic evaluations conducted to better understand the relation
between psychophysiological signals and emotions of players. This thesis is an
initiative to move away from questionnaires and physical sensors into a
non-obtrusive, user-tailored, and remote-based approach for detecting user
emotions.
}{%
% Fourth argument: filename of an image/logo for the organization
% you worked in collaboration; leave blank if not applicable
%
}% end of \cover
\end{document}
