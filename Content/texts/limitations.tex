\chapter{Limitations and critique}
\label{ch:limitations}

One potential limitation of the work presented in this thesis is the nature of the calibration games. Even though they serve the purpose of emotion elicitation materials, they were designed and developed as ordinary games. Along the process, several decisions were made concerning different aspects of each game, which inevitably affected the end result. These decisions had an impact, for instance, on the genre of each calibration game, as well as its graphical appearance and the level of complexity of the mechanics. A calibration game should induce a state of boredom at the beginning of the interaction, thus users should easily understand its mechanics in order to perceive the game as boring without a long exposure. It entails that the game mechanics must be easily understandable, preferably without much text or tutorials. Users should not spend a considerable amount of time learning the game, otherwise the concepts that induce boredom might be misunderstood and the desired emotional state would not be induced. Additionally, all calibration games should not allow users to deliberately control the mechanics' pace, since it was a key factor that was automatically controlled to induce stress towards the end of the session. Those constraints led the design of the calibration games towards more casual, 2D game mechanics. Even though games with similar characteristics exist, the proposed calibration games lack 3D content or a more complex interaction similar to those found in AAA COTS games, for instance. The genre/mechanics selected for the calibration games likely hinder several other genres and mechanics that could potentially be used as calibration games as well. The nature of the calibration games presented in this thesis does not cover the wide range of possible game types that exists, which limits its reach.

In that light, it could be argued that the calibration games proposed in this research only induced emotions elicited from the specific genres/mechanics that were selected. The use of 2D, casual foundations for the calibration games, could have conveyed a message of ``old games" to a segment of subjects/users, which would likely impact their emotional reactions. On the other hand, the use of a more complex 3D game with sophisticated mechanics, e.g. Counter Strike, is likely to require a certain level of gaming skills from participants. In such a case, it would impact the interactions of subjects that are not very familiar with gaming, which was the case for some participants in the heterogeneous groups presented in this research. As mentioned previously, individuals have different cultural views and expectations, thus, a more complex game would make it even harder to balance the design of a game with its intended purpose of inducing boredom and stress. The game Infinite Mario was used in the validation process of the proposed method mainly due to its characteristic, e.g. easy to understand and play. Additionally, it allowed more control over the content generation associated with its mechanics, therefore, boring and stressful levels could be easily developed for the experiment mentioned in the thesis. It is plausible that the emotion classification results obtained with Infinite Mario could be generalized to similar games, especially since Super Mario influenced a range of platformer games. However, as previously mentioned, the use of another 2D, casual game for the validation could limit the generalization of the results.

Another limitation of this research concerns the accuracy obtained by the method in the classification of emotional states of boredom and stress. As presented in Chapters \ref{ch:experiment2} and \ref{ch:discussion}, the method achieved an accuracy of 61.6\%. Even though this classification rate has statistical significance that proves the method performs better than random guessing, such a performance is still too low for commercial or even academic use. In its current state, the proposed method could not be used as the only tool to detect the emotional state of users, due to its noise. Additional measurements should accompany the proposed method to ensure a proper evaluation of the emotional context of subjects/users, e.g. questionnaires. However, the proposed method could still be used as an insight mechanism to analyze large amounts of video footage in an automated way, for instance. Despite the best efforts invested in this research to design an accurate emotion detector, the complexity of the task and the amount of man-power available limited the exploration process. Instead of aiming for a perfect tool, the research presented in this thesis focused on designing and rigorously evaluating each part of the proposed method. Such an approach is expected to eventually guide the construction of a more sophisticated emotion detector in the future.

It is important to highlight the technical limitations associated with the remote acquisition of physiological signals. The rPPG technique used in this research, as detailed in Section \ref{sec:experiment1-study3}, is appropriate to deal with the natural behavior that users exhibit during their interaction with games. However, this technique was likely affected by other factors not scrutinized by the research in this thesis. For instance, the 15 seconds long duration of each analysis window used for the estimation of HR may have affected the results. The ideal length of the window (called window size) is not agreed upon in the literature \parencite{rouast2016remote}. In general, it depends on the characteristics of the rPPG technique being applied as well as the hardware configuration, such as camera framerate \parencite{roald2013estimation}. The statistical nature of ICA, part of the selected rPPG employed in this research, demands longer video samples to produce accurate results. The longer the video, however, the higher the chances of subject motion, which increases noise. A trade-off between the duration of the video segments and the estimation accuracy could be better investigated. Another factor is that the experimental setup used an external light source to minimize noise caused by changes in illumination, which should narrow the estimation error to causes as subject movement and/or facial activity. Nonetheless, other factors probably could have impacted the estimation accuracy, such as facial hair, e.g. beard and fringe, use of glasses, and skin color. The results obtained with this research were achieved in a laboratory-like environment with controlled light source, which limits the generalization of the conclusions. As detailed in Chapter \ref{ch:literature-rppg}, a subject's movement and changes in illumination are significant challenges to the estimation accuracy of rPPG techniques. The use of a controlled light source, however, was deemed necessary to concentrate efforts on the remote detection of the emotional state, not on the noise caused by different illumination patterns.

Finally, a significant limitation is that the method proposed in this thesis can only detect two emotional states, i.e. stress and boredom. As detailed in Chapter \ref{ch:literature-games}, there are different models and theories about emotions. Previous works focused on emotion detection commonly classify the six basic emotions proposed by \textcite{ekman1971constants}, i.e. happiness, surprise, sadness, fear, anger and disgust. There are also a significant number of works that measure emotional states in terms of Russell's Arousal-Valence space \parencite{russell1978evidence}. The method described in this thesis relies on the emotion elicitation provoked by the concept of calibration games, which are designed to be user-tailored materials that account for the differences among users in the emotion elicitation process. These games are, by design, limited to inducing only boredom and stress. It is plausible that other emotions are also elicited by the calibration games, e.g. happiness and anger. However, the very idea of constantly and endlessly increasing the difficulty of the games to account for the subjects' individualities in the emotion elicitation process, e.g. different gaming skills and cultural expectations, directly limits the emotions that can be reasonably tracked without interrupting the gameplay. As demonstrated by the statistical tests performed in the studies conducted on the data collected from experiment 1, the emotional state of subjects is assumed to be boredom at the beginning and stress at the end of the calibration games. Inducing a state of happiness in a player, for instance, is a complex task that depends on several components, including cultural factors and gaming preferences. The same reasoning can be extended to other emotions, such as anger, fear and disgust. Even for commercial games, significant resources are invested to ensure a game is properly balanced to be able to please a wider range of users, however, there is still no guarantee that this will happen. Consequentially, the focus put on detecting only boredom and stress in the method presented in this thesis is a constraint created to counterbalance the differences that exist among subjects, regarding their different gaming, cultural and emotional profiles. However, it is important to mention that emotional states of stress and boredom are still relevant to the industry or studies in HCI and game research. Flow theory is commonly used in game research to model players emotions, which is directly connected to stress and boredom. These emotional states can be described as a function of a current player's skill level and the level of challenge he/she faces in the game. Such important emotions can help both researchers and game designers in the study of the interaction between players and games.

%One potential limitation of our work is the internal validity. As previously described, the experiment was based on a one-group posttest design, which does not use a control group to measure the effects of the treatment. Such design could be criticized for having low internal validity, since it is not possible to unambiguously attribute cause and effect \parencite{kirk1982experimental}. A two-group approach could be suggested as having stronger internal validity, since it contains a control group and allows a less ambiguous conclusion. In the context of our research, however, any multiple group design implies the comparison of physiological signals and emotional perceptions among different people. Given the social and cultural background of the participants, it is virtually impossible to compare two groups of people regarding stress/boredom. People have different preferences, culture and expectations, which cause maturation and history threats to internal validity \parencite{trochim2001research}. Additionally the process of comparing variations of physiological signals among different subjects is a complex task, even when subjects are similar, e.g. same age and sex. As a consequence, a subject in a control group might present a set of variations of signals and classify a game as boring, while a similar subject in another group might classify the same game as not boring at all, presenting a different set of variations of signals. In that light, our experiment relies on a one-group experimental design to increase internal validity, since subjects were compared with themselves, which removes inter-subject differences.
