\chapter{Games and emotions}
\label{ch:literature-games}

Research of games is a broad topic that involves different disciplines and definitions. In the context of this thesis, a game is defined as a system in which players engage in an artificial conflict, defined by rules, that results in a goal \parencite{salen2004rules}. An artificial conflict is a set of challenges, e.g. sort elements within a time constraint, the player must overcome in order to achieve the goal of the game.

\begin{figure}[h!]
    \centering
    \includegraphics[scale=0.8]{Content/figures/flow-schell.png}
    \caption{Repeating cycle of increasing challenge followed by a reward, keeping the player in the flow zone. Reproduced from \textcite{schell2014art}.}
    \label{fig:flow-schell}
\end{figure}

\textcite{schell2014art} mentions that, from a game design perspective, the difficulty level of such challenges affects the emotional state of players, e.g. moments of boredom or anxiety/stress. Every time a reward is given to the player, which is usually a tool to increase the player's power, the game's challenge level is lowered because the player becomes more skilled. After a period, this increase in the player's skill level causes the game to become boring because the challenges become easier to overcome. At that point, the game increases the difficulty level again, raising the challenge levels for the player once more, causing anxiety. The anxiety and stress period lasts until the player is rewarded again, whereupon the newly obtained power eventually lowers the challenge levels again (resulting in boredom), causing the cycle to repeat itself.

\textcite{chen2007flow} used the theory of flow or the ``theory of optimal experience'', originally established by Mihaly Csikszentmihalyi, to describe how player experience, in large part, depended on the relationship between a game's challenge and its player's level of skill. A challenge beyond the player's skill to address and overcome causes anxiety, while the opposite results in disinterest, leading to boredom \parencite{chen2007flow}. An ideal challenge/skill balance in a repeating cycle of increasing challenge followed by a reward keeps the player in a state of optimal experience and concentration, i.e. flow zone, as illustrated by Figure \ref{fig:flow-schell}.

The following sections describe in more detail the theoretical foundation of emotions, connecting them to the context of games.

%%%%%%%%%%%%%%%%%%%%%%%%%%%%%%%%%%%%%%%%%%%%%%%%%%%%%%%%%%%%%%%%%%%%%%%%%%%%%%%%%%%%%%%%%%%%%%%%%%%%%%%
\section{Emotions theory}
%%%%%%%%%%%%%%%%%%%%%%%%%%%%%%%%%%%%%%%%%%%%%%%%%%%%%%%%%%%%%%%%%%%%%%%%%%%%%%%%%%%%%%%%%%%%%%%%%%%%%%%

In the field of games research, one of the most mentioned theories regarding emotions is the theory of flow. It has been used as the foundation for several concepts, including engagement and immersion \parencite{brown2004grounded}, sense of presence \parencite{weibel2011immersion} and applicability in game design \parencite{sweetser2005gameflow, chen2007flow, cruz2017player}. Flow was originally defined as a phenomenon in which a person experiences a subjective state characterized by an intense level of attention during the execution of an intrinsically motivated activity \parencite{nakamura2014concept}. As previously mentioned, an ideal challenge/skill balance in a game leads players to an optimal state of experience and concentration, e.g. flow, therefore, flow constructs are of interest to the games community. This peculiar state of flow, however, is not limited to activities involving games; it can also be experienced in a variety of other activities, e.g. dancing and climbing. The connection of the theory of flow to contexts other than games is outside the scope of this thesis.

Further research \parencite{nakamura2014concept} refined the original definition of the flow state, culminating in the eight channel model of flow, illustrated in Figure \ref{fig:flow-eight}. This model better describes the emotional state of users according to the challenge/skill balance in relation to the subject mean, since it is more detailed and refined than the original flow model. In the eight channel model of flow, for instance, a person performing a low skill and low challenge task experiences apathy, while in the original model the classification would indicate a flow state. Emotional states such as stress and boredom can be described as a function of the current player's skill level and the level of challenge he/she is facing.

\begin{figure}[h!]
    \centering
    \includegraphics[scale=0.3]{Content/figures/flow-eight.png}
    \caption{Eight channel model of flow. Reproduced from \textcite{nakamura2014concept}.}
    \label{fig:flow-eight}
\end{figure}

Another model commonly mentioned in the literature is the basic emotions proposed by \textcite{ekman1971constants}. Constructed from an experiment involving cultural differences, it suggests that particular facial muscular patterns and discrete emotions are universal. The six emotions mentioned in the theory are happiness, surprise, sadness, fear, anger and disgust, which are strictly basic emotion models of affective state. A contrary definition is presented by \textcite{russell1978evidence}, who defined another model of emotions named Circumplex Model of Affect (CMA). Commonly referred to as Russell's Arousal-Valence (AV) space, the model is contrary to strictly basic emotion models of affective state, where each emotion emerge from independent neural systems \parencite{posner2005circumplex}. The model proposes a dimensional approach where all affective states arise from the activation of two fundamental neurophysiological systems: arousal (or alertness) and valence (a pleasure–displeasure continuum).

\begin{figure}[h!]
    \centering
    \includegraphics[scale=0.5]{Content/figures/russell-av}
    \caption{Representation of the Circumplex Model of Affect. Horizontal axis represents the valence dimension and the vertical axis represents the arousal or activation dimension. Reproduced from \textcite{posner2005circumplex}.}
    \label{fig:av-model}
\end{figure}

Figure \ref{fig:av-model} illustrates the AV space. The horizontal axis represents the valence dimension, which varies from the negative, unpleasant spectrum to the positive, pleasant spectrum. The vertical axis represents the arousal or activation dimension, which varies from low (bottom) to high (top). Each emotion is the result of a linear combination of these two dimentions. An emotional state of excitement, for instance, is conceptualized as the product of a positive activation in the neural system associated with valence along with a high activation in the neural system associated with arousal. The different scales of activation of each of those two dimensions produces different emotional states.

%%%%%%%%%%%%%%%%%%%%%%%%%%%%%%%%%%%%%%%%%%%%%%%%%%%%%%%%%%%%%%%%%%%%%%%%%%%%%%%%%%%%%%%%%%%%%%%%%%%%%%%
\section{Immersion, engagement and sense of presence}
%%%%%%%%%%%%%%%%%%%%%%%%%%%%%%%%%%%%%%%%%%%%%%%%%%%%%%%%%%%%%%%%%%%%%%%%%%%%%%%%%%%%%%%%%%%%%%%%%%%%%%%

As previously mentioned, the theory of flow is used considerably to explain emotional states and concepts in the field of games research. The definition of flow, however, requires a more sophisticated interpretation, as investigated by further research. %As previously mentioned, from a game design perspective, for instance, Schell \parencite{schell2014art} mentions that a progression of the player in the flow zone is desired, however it should not be a ``straight line'', but more of a cycle alternating among emotional states as stress, enjoyment and boredom.
Different elements are also connected to flow, such as engagement, immersion and sense of presence.

Immersion, in the concept defined by \textcite{brown2004grounded}, refers to the degree of involvement of players with a computer game. In that light, the authors theorize that a player overcomes barriers that limit his/her degree of involvement in immersion. After each barrier is broken, the sense of immersion deepens. The first barrier, for instance, is named engagement and refers to the player's willingness to invest attention and energy to learn how to play the game. The concept of flow, as defined by \textcite{nakamura2014concept}, is an extreme state, which is only achieved when the player has overcome all previously mentioned barriers and is in a ``total immersion'' state. This condition rarely happens, since it requires the player's highest level of attention. As a consequence, engagement is more plausible and common during gaming experiences than flow.

The fact that there are several works \parencite{boyle2012engagement} related to understanding and defining the concepts of engagement and immersion demonstrates the interest of researchers to broaden the view beyond flow alone. Presence, for instance, which describes the player's feeling of actually being in the game, is reported as an important aspect of engagement and immersion \parencite{weibel2011immersion}. Studies connected to training simulation \parencite{engstrom2016impact}, for instance, also indicate that contextualization (increasing the sense of presence) might affect immersion positively. As well as the efforts to understand and define engagement and immersion, there are also studies that try to measure these concepts. Most of the approaches used in those studies are based on questionnaires, which by nature players answer subjectively. Additionally, that kind of approach usually interferes with any sense of presence the player has, since it requires a shift in attention away from the game, hence breaking or affecting the level of engagement/immersion as well.

Quantitative approaches to measure engagement have also been investigated, such as the use of physiological signals, for example, heart rate \parencite{ravaja20051} and eye movement \parencite{jennett2008measuring}. The complexity in defining engagement and immersion is also reflected in the task of measuring them. \textcite{ravaja20051}, for instance, claim that the significant variation of physiological signals is an obstacle. Signals increase during emotional arousal, but decrease in response to attention engagement, which makes the measurement of engagement a non-trivial process. It highlights the difficulties in correlating qualitative data to more abstract concepts such as engagement, immersion and flow.

%Research initiatives that help in the identification of the player emotional state without breaking the current level of engagement and immersion are desired. This research aims at using the link between the ANS and emotional regulation as the foundation to analyse physiological signals of a player, which will be remotely obtained and processed in a user-tailored fashion according to data previously collected in a game-based calibration phase.

%%%%%%%%%%%%%%%%%%%%%%%%%%%%%%%%%%%%%%%%%%%%%%%%%%%%%%%%%%%%%%%%%%%%%%%%%%%%%%%%%%%%%%%%%%%%%%%%%%%%%%%
\section{Instruments for assessment of emotions}
%%%%%%%%%%%%%%%%%%%%%%%%%%%%%%%%%%%%%%%%%%%%%%%%%%%%%%%%%%%%%%%%%%%%%%%%%%%%%%%%%%%%%%%%%%%%%%%%%%%%%%%

Questionnaires are a common tool used for the assessment of emotional states of users during experiments involving games and emotions. Different formats of questionnaires are found in the literature. The simplest approach is ad-hoc questionnaires, which contain a set of questions designed by the researcher for a particular experiment. A common type of emotion scale in such an approach is Likert scales related to an emotional state, e.g. stress level.

Another instrument for the assessment of emotions is the Self-Assessment Manikin (SAM) \parencite{morris1995observations}, which is an efficient cross-cultural measurement of emotional response regarding pleasure, arousal and dominance. Figure \ref{fig:sam} illustrates the SAM mechanism. SAM employs a nonverbal, graphic depiction of three affective dimensions. In the top row of the questionnaire the user can express the level of pleasure, ranging from a figure with a smile to a figure with an unhappy face. The middle row presents figures ranging from an agitated (high arousal) state to a relaxed, sleepy state. Finally the bottom row represents the level of dominance (control) of the user in the situation at hand.

\begin{figure}[h]
    \centering
    \includegraphics[width=0.8\textwidth]{Content/figures/SAM.png}
    \caption{Visual representation of the Self-Assessment Manikin. Reproduced from \textcite{morris1995observations}.}
    \label{fig:sam}
\end{figure}

A similar approach is the Affective Slider (AS) \parencite{betella2016affective}, a digital self-reporting tool composed of two slider controls for the assessment of pleasure and arousal. Figure \ref{fig:as} illustrates the AS self-assessment mechanisms. AS also employs a nonverbal, graphic depiction of the emotional state of the user, however, the focus is on arousal and valence. A user can adjust the sliders to show his/her level of arousal (top slider) and valence (bottom slider).

\begin{figure}[h]
    \centering
    \includegraphics[width=0.7\textwidth]{Content/figures/AS.png}
    \caption{Visual representation of the Affective Slider. Reproduced from \textcite{betella2016affective}.}
    \label{fig:as}
\end{figure}

The selection of an emotion assessment instrument is connected to the research context. Both SAM and AS, for instance, are established and proven emotion measurement instruments, which would strengthen the theoretical foundations of the emotion measurement process. However, a disadvantage is that they require the researcher to instruct users on how to properly answer the questionnaire.
