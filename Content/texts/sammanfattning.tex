\chapter*{Sammanfattning}

Fr{\aa}geformul{\"a}r och fysiologiska m{\"a}tningar med hj{\"a}lp av sensorer {\"a}r i dagsl{\"a}get de vanligaste metoderna f{\"o}r insamling av data som kan anv{\"a}ndas f{\"o}r att identifiera anv{\"a}ndares k{\"a}nslotillst{\aa}nd inom m{\"a}nniska- datorinteraktion och spelforskning. Dessa metoder p{\aa}verkar dock anv{\"a}ndares naturliga beteenden d{\aa} de antingen {\"a}r p{\aa}tr{\"a}ngande under sj{\"a}lva anv{\"a}ndningstillf{\"a}llet (ex. EEG och ECG sensorer), eller genomf{\"o}rs f{\"o}rst efter anv{\"a}ndningstillf{\"a}llet. Nya metoder f{\"o}rs{\"o}ker minska den direkta p{\aa}verkan p{\aa} anv{\"a}ndaren genom att samla anv{\"a}ndardata med hj{\"a}lp av datorseende och olika fj{\"a}rrinsamlingsverktyg (ex. eye-tracking), men dessa {\"a}r f{\"o}r tillf{\"a}llet begr{\"a}nsade. Exempelvis kan m{\aa}nga av dessa metoder enbart anv{\"a}ndas i omsorgsfullt kontrollerade situationer med stimuli fr{\aa}n experimentspecifik mjukvara. F{\"o}r att m{\"a}tinstrumenten ska f{\aa} tydlig data i dessa experimentsituationer har anv{\"a}ndare ofta av n{\"o}dv{\"a}ndighet f{\"o}rh{\aa}llandevis begr{\"a}nsade interaktionsm{\"o}jligheter med specialutvecklade spel, vilket g{\"o}r det tveksamt huruvida de representerar komplexiteten hos verkliga spelsituationer. Metoderna anv{\"a}nder sig ofta {\"a}ven av projiceringsmodeller baserade p{\aa} genomsnittsdata fr{\aa}n stora anv{\"a}ndargrupper, vilket g{\"o}r att individuella egenheter hos anv{\"a}ndare ofta f{\"o}rbises. Med detta i {\aa}tanke finns det ett stort behov av nya verktyg och m{\"a}tmetoder som {\"a}r b{\aa}de icke-p{\aa}tr{\"a}ngande och anv{\"a}ndarspecifika. Denna avhandling presenterar ett forskningsprojekt d{\"a}r ett s{\aa}dant verktyg utvecklas och utv{\"a}rderas.

Det huvudsakliga kunskapsbidraget fr{\aa}n denna forskning {\"a}r en nydanande process f{\"o}r k{\"a}nslom{\"a}tning som {\"a}r icke-p{\aa}tr{\"a}ngande, anv{\"a}ndarspecifik och spelbaserad. Processen anv{\"a}nder sig av fj{\"a}rrinsamling av hj{\"a}rtrytm (HR) och r{\"o}relser i ansiktsmuskler f{\"o}r att tr{\"a}na ett anv{\"a}ndarspecifikt neuralt n{\"a}tverk som kan identifiera om anv{\"a}ndaren {\"a}r uttr{\aa}kad eller stressad. Denna l{\"o}sning {\"a}r helt automatiserad och anv{\"a}nder sig av datorseende och fotopletysmografi vid analys av videoinspelningar f{\"o}r insamling av anv{\"a}ndardata och kr{\"a}ver inga specialanpassade verktyg (ex. HR-sensorer). Processen best{\aa}r av tv{\aa} faser: en tr{\"a}nings- (eller kalibrerings-) och testningsfas. I tr{\"a}ningsfasen konstrueras och tr{\"a}nas en modell av en anv{\"a}ndares k{\"a}nslorespons under spelandet av s{\"a}rskilt utformade kalibreringsspel. Dessa kalibreringsspel {\"a}r utvecklade f{\"o}r att framkalla olika typer av k{\"a}nslorespons i form av stress och uttr{\aa}kning genom att uts{\"a}tta anv{\"a}ndare f{\"o}r utmaningar av olika sv{\aa}righetsgrader. I testfasen som f{\"o}ljer spelar anv{\"a}ndaren ett vanligt spel (ex. Super Mario). Under detta spelande fj{\"a}rrinsamlas fysiologisk anv{\"a}ndardata, vilken behandlas av den tidigare konstruerade modellen som {\"a}r anpassad f{\"o}r att tolka data fr{\aa}n just denna individuella anv{\"a}ndare. Modellen producerar slutligen en uppskattning av anv{\"a}ndarens k{\"a}nslotillst{\aa}nd under speltillf{\"a}llet.

Metoden f{\"o}r k{\"a}nslom{\"a}tning som f{\"o}resl{\aa}s i denna avhandling {\"a}r baserad p{\aa} tidigare etablerade teorier och har {\"a}ven blivit utv{\"a}rderad i en serie kontrollerade experiment. Resultatet fr{\aa}n denna utv{\"a}rdering visar att det finns en statistiskt signifikant identifiering av k{\"a}nslotillst{\aa}nd med en precision p{\aa} 61,6\%. Ut{\"o}ver presentationen av det framtagna verktyget f{\"o}r k{\"a}nslom{\"a}tningar presenteras {\"a}ven en serie av systematiska utv{\"a}rderingar av f{\"o}rh{\aa}llandet mellan psykosomatiska signaler och k{\"a}nslor. Anv{\"a}ndning av ansiktsmuskler och fysiologiska signaler (ex. HR) analyseras och deras roll som indikatorer p{\aa} k{\"a}nslotillst{\aa}nd diskuteras. Denna forskning visar att individuella egenheter i m{\"a}nniskors ansiktsuttryck kan identifieras (ex. {\"o}kad m{\"a}ngd och intensitet av olika ansiktsuttryck under stressframkallande spelsegment). Ang{\aa}ende fysiologiska signaler {\"a}r studieresultaten f{\"o}renliga med, och styrker, tidigare forskning som drar paralleller mellan HR och stressk{\"a}nslor i spelsituationer. Metoden f{\"o}r fj{\"a}rrm{\"a}tningar av k{\"a}nslotillst{\aa}nd som presenteras i denna avhandling {\"a}r brukbar, men har vissa begr{\"a}nsningar. Oavsett detta {\"a}r metoden ett lovande f{\"o}rsta steg fr{\aa}n anv{\"a}ndningen av fr{\aa}geformul{\"a}r och fysiska sensorer mot fj{\"a}rrinsamlingsbaserade l{\"o}sningar f{\"o}r utv{\"a}rderingar av anv{\"a}ndares k{\"a}nslotillst{\aa}nd.
