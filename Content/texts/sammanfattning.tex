\chapter*{Sammanfattning}

Questionnaires and physiological measurements are the most common approach used to obtain data for emotion estimation in the field of human-computer interaction (HCI) and games research. Both approaches interfere with the natural behavior of users. Initiatives based on computer vision and remote extraction of user signals for emotion estimation exist, however they are limited. Experiments of such initiatives have been performed under extremely controlled situations with few game-related stimuli. Users had a passive role with limited possibilities for interaction or emotional involvement, differently than game-based emotion stimuli, where users take an active role in the process, making decisions and directly interacting with the media. Previous works also focus on predictive models based on a group perspective. As a consequence, a model is usually trained from data of several users, which in practice describes the average behavior of the group, excluding or diluting key individualities of each user. In that light, there is a lack of initiatives focusing on non-obtrusive, user-tailored emotion detection models, in particular regarding stress and boredom, within the context of games research that is based on emotion data generated from game stimuli. This research aims to fill that gap, providing the HCI and the games research community with an emotion detection process that can be used to remotely study user's emotions in a non-obtrusive way within the context of games.

The main knowledge contribution of this research is a novel process for emotion detection that is non-obtrusive, user-tailored and game-based. It uses remotely acquired signals, namely heart rate (HR) and facial actions (FA), to create a user-tailored model, i.e. trained neural network, able to detect emotional states of boredom and stress of a given subject. The process is automated and relies on computer vision and remote photoplethysmography (rPPG) to acquire user signals, so specialized equipment, e.g. HR sensors, are not required, only an ordinary camera is needed. The approach is composed of two phases: training (or calibration) and testing. In the training phase, a model is trained using a user-tailored approach, i.e. data from a given subject playing calibration games is used to create a model for that given subject. Calibration games are a novel emotion elicitation material introduced by this research. They are games carefully designed to present a difficulty level that constantly and linearly progresses over time without a pre-defined stopping point, inducing emotional states of boredom and stress, accounting for particularities on an individual level. Finally the testing phase happens in a game session involving a subject playing any ordinary, non-calibration game, e.g. Super Mario. During the testing phase, subject's signals are remotely acquired and fed into the model previously trained for that particular subject. The model outputs the estimated emotional state of that given subject for that particular testing game.

The method for emotion detection proposed in this thesis has been conceived based on established theories and it has been carefully evaluated in experimental setups. Results show with statistical significance classification of emotional states with mean accuracy of 61.6\%. Finally, this thesis presents a series of systematic evaluations conducted to understand the relation between psychophysiological signals and emotions. Facial behavior and physiological signals, i.e. HR, are analyzed and discussed as indicators of emotional states. This research shows that individualities can be detected regarding facial activity, e.g. increased number of facial actions during the stressful part of games. Regarding physiological signals, findings are aligned with and reinforce previous research that indicates higher HR mean during stressful situations in a gaming context. Results also suggest that changes in HR during gaming sessions are a promising indicator of stress. The method presented in this thesis for remote detection of emotions is feasible, however it contains limitations. Nevertheless, it is a solid initiative to move away from questionnaires and physical sensors into a non-obtrusive, remote-based solution for evaluation of user emotions.
