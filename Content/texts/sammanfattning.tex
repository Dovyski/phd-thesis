\chapter*{Sammanfattning}

\begin{otherlanguage}{swedish}

Frågeformulär och fysiologiska mätningar med hjälp av sensorer är i dagsläget de vanligaste metoderna för insamling av data som kan användas för att identifiera användares känslotillstånd inom människa- datorinteraktion och spelforskning. Dessa metoder påverkar dock användares naturliga beteenden då de antingen är påträngande under själva användningstillfället (till exempel EEG och ECG sensorer) eller genomförs först efter användningstillfället. Nya metoder försöker minska den direkta påverkan på användaren genom att samla användardata med hjälp av datorseende och olika fjärrinsamlingsverktyg (till exempel eye-tracking), men dessa är för tillfället begränsade. Många av dessa metoder kan enbart användas i omsorgsfullt kontrollerade situationer med stimuli från experimentspecifik mjukvara. För att mätinstrumenten ska få tydlig data i dessa experimentsituationer har användare ofta förhållandevis begränsade interaktionsmöjligheter med specialutvecklade spel. Detta gör det tveksamt att de representerar komplexiteten hos verkliga spelsituationer. Metoderna använder sig även ofta av projiceringsmodeller baserade på genomsnittsdata från stora användargrupper, vilket gör att individuella egenheter hos användare ofta förbises. Med detta i åtanke finns det ett stort behov av nya verktyg och mätmetoder som är både icke-påträngande och användarspecifika. Denna avhandling presenterar ett forskningsprojekt där ett sådant verktyg utvecklas och utvärderas.

Det huvudsakliga kunskapsbidraget från denna forskning är en nydanande process för känslomätning som är icke-påträngande, användarspecifik och spelbaserad. Processen använder sig av fjärrinsamling av hjärtrytm (HR) och rörelser i ansiktsmuskler för att träna ett användarspecifikt neuralt nätverk som kan identifiera om användaren är uttråkad eller stressad. Denna lösning är helt automatiserad och använder sig av datorseende och fotopletysmografi vid analys av videoinspelningar för insamling av användardata och kräver inga specialanpassade verktyg (till exempel HR-sensorer). Processen består av två faser: en tränings- (eller kalibrerings-) och en testfas. I träningsfasen konstrueras och tränas en modell av en användares känslorespons under spelandet av särskilt utformade kalibreringsspel. Dessa kalibreringsspel är utvecklade för att framkalla olika typer av känslorespons i form av stress och uttråkning genom att utsätta användare för utmaningar med olika svårighetsgrader. I testfasen spelar användaren ett vanligt spel (till exempel Super Mario). Under detta spelande fjärrinsamlas fysiologisk användardata, vilken behandlas av den tidigare konstruerade modellen som är anpassad för att tolka data från just denna användare. Modellen producerar slutligen en uppskattning av användarens känslotillstånd under speltillfället.

Metoden för känslomätning som föreslås i denna avhandling är baserad på tidigare etablerade teorier och har även blivit utvärderad i en serie kontrollerade experiment. Resultat från utvärdering visar att det finns en statistiskt signifikant identifiering av känslotillstånd med en precision på 61,6\%. Utöver presentationen av det framtagna verktyget för känslomätningar presenteras även en serie av systematiska utvärderingar av förhållandet mellan psykofysiologiska signaler och känslor. Användning av ansiktsmuskler och fysiologiska signaler (till exempel HR) analyseras och deras roll som indikatorer på känslotillstånd diskuteras. Denna forskning visar att individuella egenheter i människors ansiktsuttryck kan identifieras (till exempel ökad mängd och intensitet av olika ansiktsuttryck under stressframkallande spelsegment). Angående fysiologiska signaler är studieresultaten förenliga med, och styrker, tidigare forskning som drar paralleller mellan HR och stresskänslor i spelsituationer. Metoden för fjärrmätning av känslotillstånd som presenteras i denna avhandling är användbar, men har vissa begränsningar. Oavsett detta är metoden ett lovande första steg bort från användning av frågeformulär och fysiskt påträngande sensorer och mot fjärrinsamlingsbaserade lösningar för utvärdering av användares känslotillstånd.

\end{otherlanguage}
