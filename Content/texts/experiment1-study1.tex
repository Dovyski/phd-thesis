\section{Study 1: variations of facial actions}
\label{sec:experiment1-study1}

This study presents information regarding FA that the subjects presented during the experiment. The 6 hours of recordings of all the subjects were analyzed manually and FA were annotated empirically. The annotations were categorized according when they occurred (the boring first part or the stressful second part of the games). An analysis of these annotated FA was conducted at group and individual level, which aimed to find patterns between the featured FA and the boring/stressful periods of the games.

The following sections presents the analysis, discussion and results of the gathered information.

\subsection{Analysis and methods}
\label{sec:experiment1-study1-methodology}

The recordings of all the subjects were analyzed by a single reseacher who took notes of any facial actions (FA) that were different from a neutral (resting) face, e.g. lips contraction, brow movement, etc. The annotations were not conducted periodically, e.g. every 5 seconds, instead they were made only when the subject's face changed from its neutral/resting state. As a consequence, if the subject's face remained in a neutral state for a long period of time, no annotations were made during that period.

This empirical and non-standard approach for facial annotation was used because the focus is not on facial expressions \textit{per se}, but in the exploration of any facial action (standardized or not) that might be used to infer patterns in boredom/stressful states. This approach is not without its limitations, however it provides a reasonable empirical perception of facial activity that is different from a neutral face, which is satisfactory for the investigation. Since FA are subtle and not necessarily part of a complete facial expression, e.g. surprise face, they might be better identified in a context where annotations are made only when facial changes happen, as opposed to a frame-by-frame analysis/annotation of a video, for instance.

\begin{figure}[!h]
\centering
\includegraphics[width=\textwidth]{Content/figures/facial-actions}
\caption{Annotated facial actions (FA). (a) Smile not showing teeth; (b) Smile showing teeth; (c) Lip pucker; (d) Lip stretcher; (e) Lip suck; (f) Lip pressor; (g) Lips parted; (h) Tongue touching lips; (i) Mouth movement right; (j) Mouth movement left; (k) Lower lip bite; (l) Frown; (m) Brow raiser; (n) Lid tightener; (o) Brow lowering}
\label{fig:fa}
\end{figure}

According to the design of the games, the subjects were supposed to experience the beginning of the games as more boring than the end, while the experience at the end was to be perceived as more stressful than the beginning of the games. As a result, if the game sessions of each subject are divided in half, in theory, one of the two resulting parts is more likely to be perceived as more boring by the subjects, while the other is more likely to be perceived as more stressful. Using that assumption, the FA annotations were divided into two groups, the ones made during the period that corresponds to the first half ($H_0$) of the games and the ones made in the second half ($H_1$). This division of the annotations aimed to identify any pattern regarding FA that occurred during periods theoretically perceived as boring or stressful. After all the annotations were made, an identification of uniqueness was performed and, based on that information, the repetitions of such unique actions across the games for all the subjects were counted. As a result, the frequency of each FA during all the game sessions was obtained, as well as when they occurred (in $H_0$ or $H_1$). Any FA that appeared just once during the entire 6 hours recorded was excluded from the list, under the assumption that such action was noise or probably part of another action. As a result, 17 unique FA that appeared in the recordings at least twice were identified. Excluding the talking and laughing FA, Figure \ref{fig:fa} illustrates all the annotated FA. Finally, after all the annotations were counted and categorized according to the period in the game, a per-subject evaluation of the frequency of FA was conducted. For each subject, an inspection was performed regarding FA that appeared with a higher frequency in $H_1$ of all three games than in $H_0$, and vice versa (appeared with a higher frequency in $H_0$ of all three games than in $H_1$).

\subsection{Results}

\begin{table}[h!]
\caption{Number of FA annotations made for all subjects during periods $H_0$ and $H_1$ of the games}
\label{table:amount-fa}
\centering
\begin{tabular}{lcc}%
\toprule%
\textbf{Game} & \textbf{Period $H_0$} & \textbf{Period $H_1$} \\
\midrule
Mushroom   & 90 & 98 \\
Platformer & 88 & 181 \\
Tetris     & 110 & 159 \\
\bottomrule%
\end{tabular}%
\end{table}

The number of subjects that featured a particular FA was analyzed, together with the number of repetitions of such FA, for all three games. The analysis also took into account the period of the game. In addition, only FA featured by two or more subjects were considered, since this analyzed more frequent FA among the whole group of subjects instead of the peculiarities of a single person. Table \ref{table:amount-fa} presents the number of FA annotations made for all the subjects during the games. According to the results, the number of FA annotations made during $H_1$ (second half) of all three games was greater than the number of annotations made during $H_0$ (first half). The increase in annotations during $H_1$ compared to $H_0$ was 8.8\%, 105.6\% and 44.5\% higher for the Mushroom, Platformer and Tetris games, respectively.

The FA annotated during each game are the following: for the Mushroom game, the three most frequent FA in $H_0$ are frown (repeated 16 times among 5 subjects), talking (12 times, 3 subjects) and tongue touching lips (9 times, 3 subjects). The three most frequent FA in $H_1$ are frown (repeated 16 times among 3 subjects), talking (13 times, 5 subjects) and lips parted (13 times, 5 subjects). A comparision of the most frequent FA in the two periods reveals that while both frown and talking are present, they are not featured by a significant number of participants. In fact, no more than 5 subjects (25\% of the participants) featured one of these FA. This suggests that individuals present distinct facial behaviors that are not easily generalizable, even in the same context. Curiously, two particular FA presented a significant change in the number of repetitions and subjects between the two periods: lip pressors (from 7 to 11 repetitions, 2 to 4 subjects) and lips parted (from 5 to 13 repetitions, 2 to 5 subjects). When compared to the whole group of participants, such an increase is not significant (again they represent less than 25\% of the participants), but it might be the indication of a pattern for two or three subjects. As suggested by previous work, the combination of such particular changes with another physiological signal, e.g. HR, might produce an acceptable detector for boredom/stress emotional states.

For the Platformer game, the three most frequent FA for $H_0$ are frown (19 repetitions among 3 subjects), tongue touching lips (12 repetitions, 3 subjects) and smile not showing teeth (11 repetitions, 3 subjects). For $H_1$, the FA are frown (49 repetitions, 5 subjects), smile not showing teeth (21 repetitions, 7 subjects) and lips parted (17 repetitions, 5 subjects). A comparison of the FA in both periods reveals that frown is featured by more subjects (5, representing 25\%) during the stressful part of the game, but more participants (7, representing 35\%) also feature smiles not showing teeth as well. In addition to these FA, 25\% of the participants feature talking behavior during $H_1$, externalizing game decisions.

For the Tetris game, the three most frequent FA for $H_0$ are frown (36 repetitions among 4 subjects), smile not showing teeth (14 repetitions, 4 subjects) and lip pressor (11 repetitions, 4 subjects). For $H_1$, the FA are frown (42 repetitions among 4 subjects), lip pressor (28 repetitions, 6 subjects) and smile not showing teeth (16 repetitions, 5 subjects). Comparing these results to the most frequent FA in the Mushroom game reveals that only frown is present in both. It is important to stress that frown is featured by less than 25\% of the participants in both games, which highlights the difficulties in finding a pattern that can be applied to all subjects, in order to identify a boring or stressful situation, even when the most frequent FA are used. On the other hand, two FA present a significant change from one period to another in the Tetris game: lip pressor (from 11 to 28 repetitions, 4 to 6 subjects) and talking (from 0 to 15 repetitions, 0 to 6 subjects). Both facial actions are featured by 30\% of the participants, which could be further investigated in the pursuit of FA that can help in the identification of emotional states. Regarding the talking FA, it has been observed in the recordings that some subjects tended to externalize in words any wrong decisions they made in the game, such as how pieces were positioned. This is similar to observations made during the Platformer game; in that sense, talking could be used as an indicator of activity in the game, since it is a clear facial manifestation that happened, in this case, when players were frustrated. For further FA analysis based on a group level, see \textcite{bevilacqua2016variations}.

Finally, a per-subject inspection of all annotated FA was conducted according to the procedure described in Section \ref{sec:experiment1-study1-methodology}. The aim was to identify, for each subject, which FA appeared in $H_0$ (or $H_1$) of \emph{all} three games with a higher frequency than they did in $H_1$ (or $H_0$), if any. Table \ref{table:individual} shows the results of this inspection. The marked numbers represent the frequency of a FA that was present in all three games for the specified subject and period. In total, 10 participants (50\%) featured at least one FA that appeared in all three games, in the same period (boring or stressful part), with a frequency equal to or greater than its appearance in the counter-period. Subject 2, for instance, featured one lip pressor during $H_0$, while the same FA appeared a total of 18 times in $H_1$ for all three games combined. It is important to highlight that subject 16 was the only one who featured a FA more frequently in $H_0$ of all three games than he/she did during $H_1$; all the other subjects featured FA more frequently in $H_1$ than in $H_0$.

\begin{table}[!h]
\caption{Subject-based frequency of FA that appeared in the same period of all three games}
\label{table:individual}
\centering
\begin{threeparttable}
\begin{tabular}{cp{.4\linewidth}cc}
\toprule%
\textbf{Subject} & \textbf{FA} & \textbf{Period $H_0$} & \textbf{Period $H_1$} \\
\midrule%
2 & Lip pressor & 1 & 18\tnote{b} \\
15 & Lip pressor & 2 & 9\tnote{b} \\
10 & Laughing & 2 & 19\tnote{b} \\
14 & Laughing & 3 & 9\tnote{b} \\
12 & Smile not showing teeth & 2 & 8\tnote{b} \\
13 & Smile not showing teeth & 0 & 6\tnote{b} \\
18 & Smile not showing teeth & 4 & 10\tnote{b} \\
11 & Lips parted & 1 & 10\tnote{b} \\
17 & Lip stretcher & 0 & 8\tnote{b} \\
16 & Talking & 7\tnote{b} & 1 \\
\bottomrule
\end{tabular}
\begin{tablenotes}
\small
\item[b]{FA was present in all three games for the specified subject and period.}
\end{tablenotes}
\end{threeparttable}
\end{table}

\subsection{Discussion}

With regard to the FA, even though further investigation is required, calculations indicate that the subjects featured a neutral face for a longer period of time during the first half ($H_0$) of all the games compared to the second half ($H_1$). Since FA annotations were made only when the subject's face featured anything different than the neutral face, more annotations indicate more facial activity. Additionally the results might indicate that the subjects featured more FA (different from the neutral face) during stressful situations than they did under boring situations, where a neutral face/expression is probably dominant.

The games used in the experiment were designed to gradually increase in the level of difficulty until the subject could no longer proceed. As a consequence, it is possible to postulate that the smiles and laughs during the second half could be connected to the subject's perception that the games were too difficult to continue playing properly. On the other hand, they could indicate genuine manifestations of enjoyment during the moments the subjects felt the game was properly balanced and engaging. Regarding other FA, such as lip pressor and lips parted, further investigation is required to accurately relate or use these actions to predict/detect emotional states. However, the results reveal a clue about how FA variations can be different at the individual level. As previously discussed, the analysis and generalization of FA at a group level is less clear than an individual approach, since FA behavior might be specific to each person. The per-subject analysis indicates that, for some of the participants, at least one FA was featured in the three games, in the same period, by the same person. Such information could be used as the starting point for further research on FA and, for example, an individual-tailored detection model for boredom/stress.

%%%%%%%%%%%%%%%%%%%%%%%%%%%%%%%%%%%%%%%%%%%%%%%%%%%%%%%%%%%%%%%%%%%%%%%%%%%%%%%%%%%%%
%\subsection{Limitations}
%%%%%%%%%%%%%%%%%%%%%%%%%%%%%%%%%%%%%%%%%%%%%%%%%%%%%%%%%%%%%%%%%%%%%%%%%%%%%%%%%%%%%

%One potential limitation of our work is the internal validity. As previously described, the experiment was based on a one-group posttest design, which does not use a control group to measure the effects of the treatment. Such design could be criticized for having low internal validity, since it is not possible to unambiguously attribute cause and effect \parencite{kirk1982experimental}. A two-group approach could be suggested as having stronger internal validity, since it contains a control group and allows a less ambiguous conclusion. In the context of our research, however, any multiple group design implies the comparison of physiological signals and emotional perceptions among different people. Given the social and cultural background of the participants, it is virtually impossible to compare two groups of people regarding stress/boredom. People have different preferences, culture and expectations, which cause maturation and history threats to internal validity \parencite{trochim2001research}. Additionally the process of comparing variations of physiological signals among different subjects is a complex task, even when subjects are similar, e.g. same age and sex. As a consequence, a subject in a control group might present a set of variations of signals and classify a game as boring, while a similar subject in another group might classify the same game as not boring at all, presenting a different set of variations of signals. In that light, our experiment relies on a one-group experimental design to increase internal validity, since subjects were compared with themselves, which removes inter-subject differences.

%Another limitation is the empirical approach used to annotate the FA, which was not based on a formal scheme and was conducted by a single person without validation by other researchers. We believe that the exploratory nature of our study regarding FA allows the use of such approach. Our aim was not to standardize FA regarding stress/boredom, but to document the perceptions of naked-eye observations of FA in a context involving games, so that it can be used to guide further steps regarding the utilization of FA in a multifactorial analysis. A frame-by-frame annotation of our video recordings using a formal scheme, such as FACS, would be a significantly laborious and time-consuming task, which is not motivated by our exploratory and empirical approach. Another limitation is the assumption used when dividing each game session in half, presuming that the middle point of the period indicates a transition from two distinct periods: $H_0$, perceived as more boring, and $H_1$, perceived as more stressing. It is not necessarily true. Even though our data indicate that subjects perceived the beginning of the games as being boring and the end as being stressful, our point of division or the periods themselves remain an assumption. There might be moments towards the end of the game, for instance, that could be perceived as more boring or joyful depending on the subject, since each participant has her/his own specific expectations and skill level regarding games. Finally the core mechanic of the Mushroom game is based on the color of the mushrooms (instead of patterns, for instance), which is not suitable for color blind subjects.

\subsection{Conclusion}

%This paper presented the description and results of an experiment aimed at exploring the variations of heart rate (HR) and facial actions (FA) during gaming sessions with induced boredom and stress. In total twenty adults of different ages and gaming experiences participated in the experiment, where they played three different games while being recorded by a video camera and monitored by a HR sensor. The games used in the experiment were carefully designed and implemented to have a difficulty level that linearly increases over time, from a boring to a stressful point. According to self-reported answers in post-games questionnaires, participants perceived the games as being boring at the beginning and stressful at the end. Such configuration gives our experiment a novel approach for the exploration of HR and FA regarding their connection to emotional states, since information can be categorized according to the induced (and theoretically known) emotional states.

The results show that more FA annotations were made during the stressful part of the games, which indicates that the participants maintained a neutral face for longer periods of time during the boring part. The analysis at group level reveals that any FA pattern is related to 5 subjects (25\% of the group) at most. In the analysis conducted at the individual level, particular patterns were found for 10 subjects (50\% of the group).

%Our findings suggest that changes in the HR during gaming sessions is a promising indicator of stress, which could be incorporated into a model aimed at emotion detection. As pointed out by previous work, a user-tailored model based on several signals, e.g. HR and FA, is more likely to detect emotional states of users. In the context where the measurement of physiological signals by physical and contact-based sensors is intrusive or not desired, e.g. remote estimation of HR, information from different channels is required. One of such additional channels of information might be facial expressions, such as the FA analysis performed in this paper. For the context of our experiment, FA analysis on an individual level produced more information to connect FA and stress/boredom emotional states. We believe that this paper contributes with information regarding HR and FA in the context of games, which can be combined to create user-tailored models for emotion detection based on different data sources.
