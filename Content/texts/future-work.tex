\chapter{Future work}
\label{ch:closing}

This thesis presented the conception, design and evaluation of several elements that are orchestrated to produce a non-obtrusive, user-tailored game-based emotion detector. Due to time and resource constraints, several courses of action were selected in favor of others. They can be further investigated to improve the proposed method or to better understand the relationship between psychophysiological signals and emotions. This chapter describes possible ideas for future work that can extend the foundations laid by this thesis.

Initially further research could be invested on the concept of calibration games. In this thesis, only three of those emotion elicitation materials were developed. As previously mentioned in the thesis, they were 2D, casual games with particular genres and mechanics. Different types of calibration games could be explored, including 3D variations in different genres, e.g. first person shooters (FPS) or strategy games. Additionally the existing calibration games proposed along with the method, i.e. Mushroom, Platformer and Tetris, could be refined and better investigated. During the debriefing sessions that followed the second experiment, several subjects mentioned their impressions regarding the calibration games. Some participants, for instance, highlighted how fast the difficulty of some games increased, e.g. Tetris. Fast increase in the difficulty level is not part of the design of any calibration game, since it is likely to induce stress on the subject in a short time period that might not be enough to be remotely analyzed, i.e. signals acquisition. The duration of each calibration game could also be further investigated. On average, subjects spent 6.4, 4.7 and 5.8 minutes playing the Mushroom, Platformer and Tetris game, respectively. No investigation was conducted regarding the ideal duration of a calibration game. Short calibration games allow quick data collection, a desirable quality to produce the user-tailored model faster. It also mitigates effects related to subject's fatigue or emotion recall when answering the questionnaires about stress/boredom at the end of each game.

Acquisition of psychophysiological signals could also be improved in many fronts. For this thesis, only two signals were used, i.e. HR and facial actions. Even though the former produces several different information about the facial, e.g. eyebrow, eye and mouth activity, more signals could be investigated. The literature review presented in this thesis found several signals that could be acquired in a remote fashion and are useful for emotion detection. The most notorious of those signals is HRV, which is widely mentioned as an indicator of stress. The addition of a new signal to the proposed method, however, requires carefully evaluation  and adaptations. Any new signal needs to be evaluated in the context of emotion elicitation, i.e. calibration games. How such signal changes in face of induced emotional states produced by the calibration games is a key aspect to be understood before it can be added to the proposed method. Following such investigation, an analysis regarding how the signal is affected by user's natural behavior is also needed. It would establish how accurately the signal can be acquired remotely, such as the evaluation of rPPG estimations of HR presented in this thesis.

The addition of any new signal to the proposed method is also intrinsically connected to further research towards the machine learning techniques used to create the user-tailored model. In this thesis, neural networks trained using random search were used, however different method could be employed. The literature mentions the use of SVM and many more machine learning models. Further research could identify better machine learning models, possibly different techniques for different users, maximizing the idea of user-tailoring the process of detecting emotions. Another possible research idea is to explore how a home environment, e.g. living room, affects user's natural behavior and how it impacts the remote estimations of the signals. As described in the thesis, all estimations of HR, for instance, were performed with an external light source, which is unlikely to exist in a home setup. A living room could housing a game console could be dark, significantly impacting the usefulness of the proposed method. Further investigations towards that topic could highlight the limitations of rPPG techniques when applied in a home environment, for instance, and how those limitations could be mitigated.

Another topic to be further explored is the differences between a user-tailored and a group model. The method proposed in this thesis advocates for a user centered design, where individualities of participants are likely to be preserved. Little investigation was conducted in this thesis regarding the use of a group approach, where data from a group of subjects is used to produce the model. A group oriented design allows the method to be trained once and used on a variety of different users without having them to play the calibration games. Further research on that front would highlight how efficient a user-tailored approach actually is compared to a group approach.

Finally this research gathered a significant amount of data related to games, psychophysiological signals and emotions from an heterogeneous group of subjects. Collected data ranges from HR information, i.e. acquired with a physical sensor, to in-game actions, e.g. jumps in Infinite Mario and movements in Tetris. Further analysis could be performed on such data to better understand the relation among in-game actions and emotions based on psychophysiological signals. Previous studies focused on relating facial activity to emotional states during interactions with Infinite Mario \parencite{shaker2011feature}. Such analysis could be further researched with the addition of physiological data.
