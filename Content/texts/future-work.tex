\chapter{Future work}
\label{ch:closing}

This thesis presents the conception, design and evaluation of several elements that are orchestrated to produce a non-obtrusive, user-tailored game-based emotion detector. Due to time and resource constraints, several courses of action were selected in favor of others. They can be further investigated to improve the proposed method or to better understand the relationship between psychophysiological signals and emotions. This chapter describes possible ideas for future work that can extend the foundations laid by this thesis.

Initially, further research could be invested on the concept of calibration games. In this thesis, only three of those emotion elicitation materials were developed. As previously mentioned, they were 2D, casual games with particular genres and mechanics. Different types of calibration games could be explored, including 3D variations in different genres, e.g. first person shooters (FPS) or strategy games. Additionally the existing calibration games proposed along with the method, i.e. Mushroom, Platformer and Tetris, could be refined and better investigated. During the debriefing sessions that followed the second experiment, several subjects mentioned their impressions regarding the calibration games. Some participants, for instance, highlighted the speed at which the difficulty of some games increased, e.g. Tetris. A fast increase of the difficulty level is not part of the design of any calibration game, since it is likely to induce stress in the subject in a short time period which might be insufficient for remote analysis, i.e. signals acquisition. The duration of each calibration game could also be further investigated. On average, subjects spent 6.4, 4.7 and 5.8 minutes playing the Mushroom, Platformer and Tetris game, respectively. No investigation was conducted regarding the ideal duration of a calibration game. Short calibration games allow quick data collection, a desirable quality for the speedy production of the user-tailored model. It also mitigates effects related to a subject's fatigue or emotion recall when answering the questionnaires about stress/boredom at the end of each game.

The acquisition of psychophysiological signals could also be improved on many fronts. For this thesis, only two signals were used, i.e. HR and facial actions. Even though the former produces different information about the face, e.g. eyebrow, eye and mouth activity, more signals could be investigated. The literature review presented in this thesis found several signals that could be acquired in a remote fashion and used for emotion detection. The most notorious of those signals is HRV, which is widely mentioned as an indicator of stress. Another possible signal is related to the eye, including eye tracking, saccade time and blinking. Even though eye tracking is strongly connected to the activity at hand, e.g. movement pattern of the eyes is highly correlated with the game mechanics being played, while blinking and eye saccade time are less influenced by the game mechanics. Consequentially, they could easily be integrated and used as emotional indicators. However, the addition of a new signal to the proposed method requires careful evaluations and adaptations. Any new signal needs to be evaluated in the context of emotion elicitation, i.e. calibration games. How such a signal changes in face of induced emotional states produced by calibration games is a key aspect to be understood before it can be added to the proposed method. Following such investigation, an analysis regarding how the signal is affected by a user's natural behavior is also needed. It would establish the accuracy level of the signal's acquisition, such as the evaluation of rPPG estimations of HR presented in this thesis.

The addition of any new signal to the proposed method is also intrinsically connected to further research on the machine learning techniques used to create the user-tailored model. In this thesis, neural networks trained using random search were used, however different methods could be employed. The literature mentions the use of SVM and many more machine learning models. Further research could identify better machine learning models, possibly different techniques for different users, maximizing the idea of user-tailoring the process of detecting emotions. Another possible research idea is to explore how a home environment, e.g. living room, affects a user's natural behavior and how it impacts the remote estimations of the signals. As described in the thesis, all estimations of HR, for instance, were performed with an external light source, which is an unlikely home setup. A living room containing a game console could be dark, significantly impacting the usefulness of the proposed method. Further investigations of that topic could highlight the limitations of rPPG techniques when applied in a home environment, for instance, and how those limitations could be mitigated.

Another topic for further exploration is the difference between a user-tailored and a group model. The method proposed in this thesis advocates a user centered design, where individualities of participants are likely to be preserved. In this thesis, little research was invested on the use of a group approach, where data from a group of subjects are used to produce the model. A group-oriented design allows the method to be trained once and used on a variety of different users without them having to play the calibration games. Further research on that front would highlight how efficient a user-tailored approach actually is compared to a group approach. One example of such an investigation is the middle ground between a user- and a group-tailored approach. In such an approach, calibration games are used as emotion elicitation material to enhance the readings of individual traits when training a group model. Calibration games are designed to elicit particular emotional states at an individual level, it is therefore expected that a group model trained on several of those enhanced individual manifestations will produce a better emotion classifier. Works in the literature commonly use the same emotion stimuli on all subjects, e.g. an invariant image or game, consequentially subjects may perceive that in an emotionally differently way, e.g. some subjects might not be affected at all. These subjects might not contribute to the training of a group model, because their emotional perception was not properly captured. Observing and capturing such individualities is precisely the aim of a calibration game, which could help produce better group models.

Finally, this research gathered a significant amount of data related to games, psychophysiological signals and emotions from a heterogeneous group of subjects. This data range from HR information, i.e. acquired with a physical sensor, to in-game actions, e.g. jumps in Infinite Mario and movements in Tetris. Further analysis could be performed on such data to better understand the relation between in-game actions and emotions based on psychophysiological signals. Previous studies focused on relating facial activity to emotional states during interactions with Infinite Mario \parencite{shaker2011feature}. Such analysis could be further researched with the addition of physiological data, for instance. Additionally, the data collected by this research can be segmented according to the gaming profile of subjects, e.g. categorize subjects according to reported gaming skills or weekly hours devoted to gaming. This could produce insights regarding the differences between casual and ``hardcore" players in relation to changes in psychophysiological signals, accuracy of remotely detecting emotional states, body behavior and self-reported levels of stress and boredom.
