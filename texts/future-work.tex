\chapter{Future work}
\label{ch:closing}

This thesis presented the design, conception and evaluation of several elements that are orchestrated to produce a non-obtrusive, user-tailored game-based emotion detector. Due to time and resource constraints, several courses of action were selected in favor of others, however they can be further investigated to improve the proposed method or to better understand the relationship between psychophysiological signals and emotions. This chapter describes possible ideas for future work that can extend the foundations laid by this thesis.

Further research could be invested on the concept of calibration games. In this thesis, only three of those carefully designed emotion elicitation materials were developed. As previously mentioned in the thesis, they were 2D, casual games with particular genres and mechanics. Different types of calibration games could be explored, including 3D variations in different genres, e.g. first person shooters (FPS) or strategy games. Additionally the existing calibration games proposed along with the method, i.e. Mushroom, Platformer and Tetris, could be refined and better investigated. During the debriefing sessions that followed the second experiment, several subjects mentioned their impressions regarding the calibration games. Some participants, for instance, highlighted how fast the difficulty of some games increased, e.g. Tetris. Fast increase in the difficulty level is not part of the design of any calibration game, since it is likely to induce stress on the subject in a short time period that might not be enough to be captured by the remote readings. The duration of each calibration game could also be further investigated. On average, subjects spent 6.4, 4.7 and 5.8 minutes playing the Mushroom, Platformer and Tetris game. No investigations were conducted regarding the ideal duration of a calibration game. Ideally a short calibration game that allows the collection of quality data is desirable, since the user-tailored model can be produced faster. It also mitigates effects related to subject's fatigue or emotion recall when answersning the questionnaires about stress/boredom levels in the games.

Acquisition of psychophysiological signals could also be improved in many fronts. For this thesis, only two signals were used, i.e. HR and facial actions. Even though the former produces several different information about the facial, more signals could be investigated. The literature review presented in this thesis mentions several signals that are likely to be acquired in a remote fashion. The most notorious of those signals is HRV, which is widely mentioned as an indicator of stress. The addition of a now signal into the proposed method, however, requires a set of investigations and adaptations. Any new signals needs to be evaluated in the context of the emotion elicitation, i.e. calibration games. How such signals changes in face of the induced emotional states produced by the calibration games is a key aspect to be understood before any new signal is added. Following such investigation, an analysis regarding how the signal is affected by user's natural behavior is needed. It would establish how accurately the signal can be acquired remotely, such as the evaluation of rPPG estimations of HR presented in this thesis. The addition of a new signal is also connected to further research towards the training process of the user-tailored model. In this thesis, neural networks trained using random search were used, however different method could be employed. The literature mentions the use of SVM and many more machine learning models. Further research could identify better machine learning models, possibly different techniques for different users, maximizing the idea of user-tailoring the process of detecting emotions. Another possible research idea is to explore how the illumination of a home environment, e.g. living room, and user's natural behavior impacs the remote estimations of HR. As described in the thesis, all estimations of HR were performed with an external light source, which is unlikely to exist at home. Further investigations towards that topic could highlight the limitations of rPPG techniques when applied in a home environment, and how those limitations could be mitigated.

Finally this research gathered a significant amount of data from an heterogeneous population. Collected data includes HR information, i.e. acquired with a physical sensor, in-game actions, e.g. jumps in Infine Mario and movements in the Tetris game. Further analysis could be performed on such data to better understand the relation among in-game actions, psychophysiological signals and emotions. Previous studies focused on relating facial activity to emotional states during interactions with Infinie Mario \parencite{shaker2011feature}. Such analysis could be further researched with the addition of physiological data. Another aspect is to test



%The tasks involve the refinement of the process of remote acquisition of signals, definition of inputs for the user-tailored model, investigation of machine learning techniques, execution of an experiment involving emotion detection and finally the instantiation of the proposed method as a software tool. Table \ref{tab:schedule} illustrates the schedule regarding the progression of the tasks. The following sections describe the tasks in detail.
