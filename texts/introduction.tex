\chapter{Introduction}
\label{c:introduction}

In human-computer interaction (HCI) research, the study of the relation between users and systems is of interest. Within the context of games research in particular, the relation between the player and the game is an important topic. Such relation comprehends concepts as engagement and immersion \parencite{boyle2012engagement} and the investigation of the elements that influence such concepts.
%Presence, for instance, which describes the player's feeling of actually being in the game, is reported as an important aspect of engagement and immersion \parencite{weibel2011immersion}. Studies connected to training simulation \parencite{engstrom2016impact} also indicate that contextualization (increasing the sense of presence) might affect immersion positively.
To perform such investigations, researchers need to rely on methods that are able to capture the user's state within the proposed context. Questionnaires about the user's emotional state are common tools used in the process, e.g. to obtain information about stress and boredom levels.
%Questionnaires, however, require the user to stop the game activity in order to share his/her current state. The frequency that such questionnaires are issued is also a concern. If performed too often, more information might be collected, but the data might contain noise caused by the frequent interruptions, e.g. player is more stressed/bored by the questionnaire interruptions than by the game itself. If performed too sparse, not enough information will be gathered from the player.
As a side effect, however, questionnaires require a shift in attention, hence breaking or affecting the level of engagement/immersion of the user. As an addition to (or replacement of) questionnaires, physiological signals have been used to obtain information from users without causing interruptions \parencite{bousefsaf2013remote,yun2009game,rani2006empirical,tijs2008dynamic}. Heart rate (HR), for instance, is a source of information to measure emotional states \parencite{kivikangas2011review}, which can be used to detect emotions such as stress \parencite{choi2009using} or boredom \parencite{yamakoshi2007preliminary}.
%In games research, quantitative approaches already used HR to measure engagement \parencite{ravaja20051}, for instance.
%there are initiatives to measure such states and other elements, such as engagement/immersion \parencite{boyle2012engagement} and presence \parencite{weibel2011immersion}.

Computer games were proved to provoke alteration in the mean HR of players at stressful periods of gameplay \parencite{sharma2006assessment,rodriguez2015vr}. A common approach to obtaining HR measurements is the use of physical sensors attached to a player. They allow a continuous measurement of the signal, however they are intrusive and might restrict player's motion abilities, e.g. a sensor attached to a finger prevents the use of that finger. The intrusive approach also increases user's awareness of being monitored \parencite{yamakoshi2007preliminary,yamaguchi2006evaluation,healey2005detecting}, which disturbs the results of any game research investigation being performed.
As an alternative, investigations \parencite{mcduff2015survey} have shown that it is possible to remotely measure HR by analyzing a video of a subject using remote photoplethysmography (rPPG) \parencite{allen2007photoplethysmography}. The pressure of the cardiac activity causes the blood vessels to change their volume (because of the pulse), which makes the light absorption on the skin surface change accordingly. PPG is a time-varying signal resulted from such differences in the light absorption in live human tissue, which can be processed to calculate the HR. The remote detection of HR proved a promising approach to infer boredom/stress levels \parencite{kukolja2014comparative} or cognitive stress \parencite{mcduff2014remote} of a person. Experiments regarding such approaches, however, were performed under extremely controlled situations with few game-related stimuli. A significant limitation of such approaches was that subjects were asked to remain still during the experiment. Another problem is that subjects had limited interaction with the content being presented: they performed tasks mentally (e.g. counting), watched videos/images or performed gamified cognitive tests for a short period of time. Those are artificial situations that are unlikely to happen in real-life situations, especially in a gaming session with a challenging game lasting for several minutes. In that situation, the subject will probably move and present variations of facial actions during the gaming session \parencite{bevilacqua2016variations}.

rPPG-based methods for HR measurement are tools that can be used by the HCI community, particularly in games research, however its reliability within a context involving natural behavior must be checked. The methods are sensitive to noise caused by movement, facial expressions or changes in illumination (e.g. screen activity reflected on user's face), which are all likely to happen in such sessions with natural behavior. Those interferences might produce unreliable measurements of the HR signal, resulting in misleading investigations. It is important to establish the reliability of remote HR measurements under situations with natural behavior, where users are not instructed to behave differently than what they usually do. In that light this paper presents an analysis of the remotely obtained HR signal of users within such natural context. We developed a set of casual-themed, similar to off-the-shelf games that were carefully designed to present stressful and boring moments, which should induce players to present variations of HR. During the gaming sessions, the player's HR signal was remotely estimated using the work by \textcite{poh2011advancements}, an established rPPG technique for HR estimation. A physical sensor was used as ground truth. A comparison and analysis of the accuracy of remote HR estimations are presented and discussed. The main contribution of this paper is the accuracy evaluation of an established rPPG technique within the context of gaming sessions where users behave naturally instead of following movement constraint rules, e.g. remain still. Our results provide researchers with information related to the reliability of a remote HR measurement technique when applied to contexts where users behave more naturally.

%%%%%%%%

Research in different areas, such as affective computing and computer vision, aim to improve the current detection workflow with non-obtrusive approaches. By using computer vision and a video stream captured by a camera, one can obtain different information from a subject, such as facial expressions and physiological signals, without the use of physical sensors. The analysis of such signals combined, know as multimodal analysis, is a promising approach to infer a person's boredom/stress levels \parencite{kukolja2014comparative}.

This research presents an approach for remotely detecting and measuring the changes in stress and boredom levels of a player during the interaction with a game. The process is based on a non-contact multimodal analysis of user signals, e.g. heart rate and facial expressions, obtained from a video stream. The foundation is based on the construction of a user-tailored model, built on information obtained from the user while he/she played a set of calibration games. The model uses the remotely obtained signals from the user in conjunction with the calibration data to detect the player's changes regarding stress and boredom levels in any other game.


\section{Motivation}

\section{Problem specification}

\section{Contributions}

\section{Thesis overview and structure}
