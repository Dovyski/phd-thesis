\chapter{Introduction}
\label{c:introduction}

\section{Motivation and problem specification}

In human-computer interaction (HCI) research, the study of the relation between users and systems is of interest. Within the context of games research in particular, the relation between the player and the game is an important topic. Such relation comprehends concepts as engagement and immersion \parencite{boyle2012engagement} and the investigation of the elements that influence those concepts.

To perform such investigations, researchers need to rely on methods that are able to capture the user's state within the proposed context. The most commonly used techniques to obtain data regarding emotional states of players in a game are self-reports (questionnaires) and physiological measurements \parencite{mekler2014systematic}. Questionnaires are practical and ease to use tools, however they require a shift in attention, hence breaking or affecting the level of engagement/immersion of the user. Physiological signals, on the other hand, have been used to obtain information from users without causing interruptions \parencite{bousefsaf2013remote,yun2009game,rani2006empirical,tijs2008dynamic}. Sensors, despite avoiding interruptions, are usually perceived as uncomfortable and intrusive, since they require the proper setup in the person's body. Additionally sensors might restrict player's motion abilities (e.g. a sensor attached to a finger prevents the use of that finger) as well as increase user's awareness of being monitored \parencite{yamakoshi2007preliminary,yamaguchi2006evaluation,healey2005detecting}, which disturbs the results of an investigation.

%Questionnaires, however, require the user to stop the game activity in order to share his/her current state. The frequency that such questionnaires are issued is also a concern. If performed too often, more information might be collected, but the data might contain noise caused by the frequent interruptions, e.g. player is more stressed/bored by the questionnaire interruptions than by the game itself. If performed too sparse, not enough information will be gathered from the player.

There is a significant amount of information that can be read from a human body over time, such as heart rate (HR), respiratory rate (RR) and facial expressions, among others, which can be seen as input signals for emotion estimation. The process of mapping such signals to an emotional state, however, is a significant task. It involves testing/defining what are the possible emotional states a person can experience \cite{mandryk2006continuous}, as well as comparing which signals are better predictors of such states \cite{jerritta2011physiological}. A number of studies \cite{kukolja2014comparative} suggest that the analysis of a combination of different input signals, known as multimodal analysis, is more likely to produce accurate results when mapping emotional states. The mapping process itself is another important part of the equation that influences the results.

Research in different areas, such as affective computing and computer vision, aim to improve the workflow of emotion investigation with non-obtrusive approaches involving the aforementioned signals. By using computer vision and a video stream captured by a camera, one can obtain different information from a subject, such as facial expressions and physiological signals, without the use of physical sensors. In that light, this research presents an approach for remotely detecting and measuring the changes in stress and boredom levels of a player during the interaction with a game. The process is based on a non-contact multimodal analysis of user signals, e.g. HR and facial expressions, obtained from a video stream. The foundation is based on the construction of a user-tailored model, built on information obtained from the user while he/she played a set of calibration games. The model uses the remotely obtained signals from the user in conjunction with the calibration data to detect the player's changes regarding stress and boredom levels in any other game.

\section{Contributions}

This research presents a method that is able to interpret remotely acquired signals from a person and detect his/her current emotional state regarding stress and boredom according to data obtained from a calibration phase. The result adds to the body of knowledge of HCI and games research. The main contribution is a method for remote acquisition and analysis of physiological and non-physiological signals of a player in order to detect the stress and boredom levels of that player.

A purely remote-based approach, as the one proposed by this research, enhances the tooling available regarding investigation methods of stress and boredom. The approach, which is based on calibration games, maps a set of variations of signals into two specific emotional states (stress/boredom). This information can be used by other researchers to identify important moments during the interaction of players with games, such as when the recognized pattern is closer to stress. In game design research, for instance, that instrumentation can be used as another way of obtaining information from a user during a game session. The use of questionnaires, which shift the player's focus away from the game, can be enhanced and/or replaced by the use of the proposed method, making the process less obtrusive. By remotely reading information regarding stress and boredom, a researcher can use such information to better understand concepts as engagement, frustration, immersion and flow in games, for instance. Additionally it can be used in any activity that relies on stress/boredom as an important measurement, such as usability tests in software and games, for instance. Another contribution is a better understanding of how the selected signals are related to stress/boredom. Other researchers might use that information in contexts outside the games area, such as the measurement of costumers satisfaction or interest in stores.

%The proposed method will be based on the analysis of the variation of signals of a person according to a reference (calibration data). This approach is significantly different from the already existing methods, which focus on training a model to detect pre-defined states (e.g. stress, boredom, rest) based on the current value of the acquired signals. The foundation of the approach proposed in this research is the use of variations, which by nature account for differences between the current state and any other known state (the calibration profile, for instance). This configuration allows the method to be expanded or further investigated to produce a scale regarding the measurement of stress and boredom. Different from a direct mapping of signals to a state, a scaled measurement might inform the researcher of how much stress or boredom the player is experiencing, as opposed to just informing he/she is stressed or bored. This might be possible to be achieved with a machine learning model, for instance, but it will require a complex training setup. which will inevitably result in frequent interruptions of the player for collecting self-reported stress/boredom levels. This will disturb immersion/engagement with the game, probably resulting in noise in the mapping.

%A researcher is be able to increase the internal validity of his/her workflow by ensuring the player keeps the focus on the game without interruptions and by minimizing the side effects (and inconveniences) of physical monitoring. This research can also be deployed as a solution for game developer studios to automatically analyze hours of recorded gameplay and highlight the moments when boredom/stress levels changed significantly. As a complement the solution is based on a single, ordinary camera and a software implementation, which eliminates the use of complex setups of physical sensors. It eases the investigation process and reduces costs.

\section{Thesis overview and structure}
