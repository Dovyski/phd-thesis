\chapter{Research methodology}

The aim of this research is to produce a model (and a software that uses it) to infer the emotional state of a player regarding stress and boredom. Since the result of the research is a model, which will be built from different measurements to predict or infer a state, the present work stands on the positivism paradigm. Essentially this work will formulate a theory about how the variation of physiological signals relate to stress/boredom levels in the context of games and how it can be detected. The involvement of humans in the process might relate to the social side of interpretivism, however the foundation of the work is still based on the analysis of physiological signals. Such signals and their patterns might be different for each person, however they are still ordered and regular under the human being perspective. As a consequence, they can be objectively observed, measured and analysed with a quantitative approach and hypothesis testing.

The strategy used to build and validate the knowledge in this project is experimental research. Such approach is composed of a set of research designs that use controlled testing and manipulation of variables in order to understand casual processes. The foundation of an experiment is to manipulate a variable (or a set of them) and measure any changes in other variables. It establishes the effect on a dependent variable, which is the focus of the research. The model being constructed in this research links the variations of physiological signals to stress/boredom levels in the context of games, hence there is a causal effect in the process since identified variations (cause) will precede changes in stress/boredom levels (effect). It progresses to the construction of a hypothesis where the cause will consistently lead to the same effect, at least for the same person, so the link between variations of signals and emotional levels can be inferred or predicted.

\section{Research objectives}

This research is expected to produce a method that is able to interpret remotely acquired signals from a person and detect his/her current emotional state regarding stress and boredom according to data obtained in a calibration phase. The model will be implemented in a software that uses a video feed to detect the person's emotional state. %This software could be used in games research, as well as a testing tool for the evaluation of stress/boredom during the development of commercial games (in play-testing sessions, for instance).

The following research objectives (O) have being identified to support the overall aim of this thesis:

\textbf{O1}: identify the main concepts, theories and signals associated with the psychophysiological profile of users and their emotions within the field of HCI, particularly regarding games research. The result of this objective is a formal definition of stress and boredom within the context of this research, as well as the identification of which physiological and non-physiological signals are commonly applied to emotion detection.

\textbf{O2}: identify the ideal existing computer vision techniques that can be employed to remotely extract the identified physiological and non-physiological signals of users via analysis of videos. The investigation includes the analysis of how existing techniques are being applied to emotion detection. The set of signals to be remotely extracted is based on the results of O1.

\textbf{O3}: investigate the feasibility, accuracy and challenges of applying the identified computer vision techniques regarding the extraction of the signals when applied to the context of computer games. This objective comprehends the analysis of the behavior of players during gaming sessions and how it affects the technique.

\textbf{O4}: investigate and validate the concept of a game-based calibration phase as an emotional elicitation source able to provide data to fit a user-tailored predictive model. The result of this objective is to design and validate a set of calibration games that can trigger the emotional responses required for the analysis of the remotely obtained signals and detection of boredom/stress levels by the model.

\textbf{O5}: Propose and validate a user-tailored, multifactorial model that uses the identified physiological and non-physiological signals, the computer vision technique and the calibration data to detect the current stress/boredom levels of a person while he/she plays any video game.

%The objective of this research is to produce a method that is able to interpret remotely acquired signals from a person playing a game and detect his/her current emotional state regarding stress and boredom according to data previously obtained in a calibration phase. The model will be implemented in a software that uses a video feed to detect the person's emotional state.

%The current approaches used to obtain information from the players during games research inevitably affect the player's experience. They require the user to stop the game activity in order to share his/her current state, such as by answering a questionnaire. The frequency that such questionnaires are issued is also a concern. If performed too often, more information might be collected, but the data might contain noise caused by the frequent interruptions, e.g. player is more stressed/bored by the questionnaire interruptions than by the game itself. If performed too sparse, not enough information will be gathered from the player. A physical sensor attached to a player, on the contrary, allows a continuous monitoring process, however it is intrusive and might interfere with the player capacity to interact with the game. It might prevent the use or movement of specific parts of the body, for instance. Physical sensors also increase the chances of the player to behave differently as a side-effect of the monitoring process itself.

%A purely remote-based solution, as the one proposed by this research, enhances the tooling available to the games research community regarding investigation methods of stress and boredom. A games researcher will be able to increase the internal validity of his/her workflow by ensuring the player keeps the focus on the game without interruptions and by minimizing the side effects (and inconveniences) of physical monitoring. This research could also be deployed as a solution for game developer studios to automatically analyze hours of recorded gameplay and highlight the moments when boredom/stress levels changed significantly. As a complement the solution will be based on a single, ordinary camera and a software implementation, which eliminates the use of complex setups of physical sensors. It eases the investigation process and reduces costs.

\section{Research process}

The experiment design will be based on a within-subject approach \cite{lane2015online}. In such approach, all participants perform at all levels of the treatment and there are no control groups. It is the opposite of a between-subjects approach, where subjects are divided in more than one group that receive different treatments. In that approach there are special groups, called control groups, that receive no treatment. The comparison between the control groups and the treatment groups ensures internal validity. In the context of this research, physiological signals will be measured, so the division of subjects into more than one group poses a comparison problem. Each individual will inevitably differ from one another regarding physiological signals, such as variations in average HR during rest, for instance. When measuring HR, for instance, some subjects will have higher/lower HR mean than others, independent of the group they are in or the treatment they undergo. To counter that problem, the experiment will use a one-group posttest design \cite{kirk1982experimental}, as illustrated by Figure \ref{fig:experiment}. Using the first row as an example, subject $S_0$ played game $G_a$ as the first level of the treatment, followed by a post-test of that game ($PT_a$), then a rest period. In the second level of the treatment, the subject played game $G_b$, followed by a post-test of that game ($PT_b$), then another rest period. Finally in the third level of the treatment, the subject played game $G_c$ followed by a post-test of that game ($PT_c$).

\begin{figure}[ht]
    \centering
%    \includegraphics[scale=0.5]{imgs/experiment_design.png}
    \caption{One-group posttest experiment design used in this research. $S_j$ represents the $j^{\text{th}}$ subject, $G_i$ represents a game of type $i$, $PT_i$ is the post-test for game $G_i$ and $rest$ is a resting period.}
    \label{fig:experiment}
\end{figure}

By using a one-group posttest design, each individual will perform on all levels of the treatment (play a set of different games). The within-subjects approach ensures that the differences between subjects are not interfering in the comparison, since a subject is being compared to his/herself in the different levels of the treatment. Subjects are not being compared among each other. In essence, each subject is serving as his/her own control group. According to Kirk \cite{kirk1982experimental}, the one-group posttest design should only be used when the researcher knows the mean value of the independent variable when no treatment is in effect. Such information will be obtained during the resting periods of the experiment, where the baseline value for all measured signals can be established for each subject.

The process of sampling a group of participants for each experiment will follow the convenience sampling approach, a non-probability sampling technique where participants are recruited because of their convenient accessibility/proximity to the researcher. Volunteers will be randomly recruited for each experiment. A probability sampling approach, where each individual of the population has an equal chance of being selected, would be ideal and would strength the external validity of the research. However the costs, logistics and time constraints associated with it makes such approach impractical in the context of this research.

\section{Delimitation}

\section{Current state of research}
