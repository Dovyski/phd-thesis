\chapter*{Abstract}

Questionnaires and physiological measurements are the most common approach used to obtain data for emotion estimation in the field of human-computer interaction (HCI) and games research. Both approaches interfere with the natural behavior of users, which affects any research procedure. Initiatives based on computer vision and remote extraction of user signals for emotion estimation exist, however they are limited. Experiments of such initiatives have been performed under extremely controlled situations with few game-related stimuli. Users had a passive role with limited possibilities for interaction or emotional involvement, differently than game-based emotion stimuli, where users take an active role in the process, making decision and directly interacting with the media. Previous works also focus on predictive models based on a group perspective. As a consequece, a model is usually trained from data of several users, which in practice describes the average behavior of the group, excluding or diluting key individualities of each user. In that light, there is a lack of initiatives focusing on non-obtrusive, user-tailored emotion detection models, in particular regarding stress and boredom, within the context of games research that are based on emotion data generated from game stimuli.

This thesis proposal presents a research that aims to fill that gap, providing the HCI and the games research community with an emotion detection process, instantiated as a software tool, which can be used to remotely study user's emotions in a non-obtrusive way within the context of games. The main knowledge contribution of this research is a novel process for emotion detection that is remote (non-contact) and constructed from a game-based, multifactorial, user-tailored calibration phase. The process relies on computer vision and remote photoplethysmography (rPPG) to read user signals, e.g. heart rate (HR) and facial actions, without physical contact during the interaction with games in order to perform the detection of stress/boredom levels of users. The approach is automated and uses an ordinary camera to collect information, so specialized equipments, e.g. HR sensors, are not required.

Current results of this research show that individualities can be detected regarding facial activity, e.g. increased number of facial actions during the stressful part of games. Regarding physiological signals, findings are aligned with and reinforce previous research that indicate higher HR mean during stressful situations in a gaming context. The findings also suggest that changes in the HR during gaming sessions is a promising indicator of stress, which can be incorporated into a model aimed at emotion detection. The literature reviews, the experiments conducted so far and the planned future tasks support the idea of using a set of signals, e.g. facial activity, body movement, and HR estimations as sources of information in a multifactorial analysis for the identification of stress and boredom in games. It will produce a novel user-tailored approach for emotion detection focused on the behavioral particularities of each user instead of the average group pattern. The proposed approach will be implemented as a software tool, which can be used by researchers and practitioners for games research.
