\chapter{Experiment 2: validation of remote detection of emotions}
\label{ch:experiment2}

\section{Introduction}
This chapter details the software developed to be an instanciation of the artifact produced as the outcome of this research, which is an approach for remotely detecting the changes in stress and boredom levels of a player during the interaction with a game.

\section{Method}
  \subsection{Experimental setup}
  \subsection{Experimental procedure}
    \subsubsection{Calibration games}
    \subsubsection{Evaluation game}
    \subsubsection{Data collection}
  \subsection{Data preprocessing}
  \subsection{Emotion estimation}
    \subsubsection{Relevant feature extraction}
    \subsubsection{Emotion classification}
    \subsubsection{Predicting emotions}
\section{Results}
  \subsection{Self-reported emotional state}
  \subsection{Emotion classification}
\section{Discussion}
\section{Conclusion}

%The experiment design will be based on a within-subject approach \cite{lane2015online}. In such approach, all participants perform at all levels of the treatment and there are no control groups. It is the opposite of a between-subjects approach, where subjects are divided in more than one group that receive different treatments. In that approach there are special groups, called control groups, that receive no treatment. The comparison between the control groups and the treatment groups ensures internal validity. In the context of this research, physiological signals will be measured, so the division of subjects into more than one group poses a comparison problem. Each individual will inevitably differ from one another regarding physiological signals, such as variations in average HR during rest, for instance. When measuring HR, for instance, some subjects will have higher/lower HR mean than others, independent of the group they are in or the treatment they undergo. To counter that problem, the experiment will use a one-group posttest design \cite{kirk1982experimental}, as illustrated by Figure \ref{fig:experiment}. Using the first row as an example, subject $S_0$ played game $G_a$ as the first level of the treatment, followed by a post-test of that game ($PT_a$), then a rest period. In the second level of the treatment, the subject played game $G_b$, followed by a post-test of that game ($PT_b$), then another rest period. Finally in the third level of the treatment, the subject played game $G_c$ followed by a post-test of that game ($PT_c$).

%\begin{figure}[ht]
%    \centering
%    \includegraphics[scale=0.5]{imgs/experiment-design.png}
%    \caption{One-group posttest experiment design used in this research. $S_j$ represents the $j^{\text{th}}$ subject, $G_i$ represents a game of type $i$, $PT_i$ is the post-test for game $G_i$ and $rest$ is a resting period.}
%    \label{fig:experiment}
%\end{figure}

%By using a one-group posttest design, each individual will perform on all levels of the treatment (play a set of different games). The within-subjects approach ensures that the differences between subjects are not interfering in the comparison, since a subject is being compared to his/herself in the different levels of the treatment. Subjects are not being compared among each other. In essence, each subject is serving as his/her own control group. According to Kirk \cite{kirk1982experimental}, the one-group posttest design should only be used when the researcher knows the mean value of the independent variable when no treatment is in effect. Such information will be obtained during the resting periods of the experiment, where the baseline value for all measured signals can be established for each subject.

%The process of sampling a group of participants for each experiment will follow the convenience sampling approach, a non-probability sampling technique where participants are recruited because of their convenient accessibility/proximity to the researcher. Volunteers will be randomly recruited for each experiment. A probability sampling approach, where each individual of the population has an equal chance of being selected, would be ideal and would strength the external validity of the research. However the costs, logistics and time constraints associated with it makes such approach impractical in the context of this research.
