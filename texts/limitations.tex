\chapter{Limitations and critique}
\label{ch:limitations}

One potential limitation of the work presented in this thesis is the nature of the calibration games. Even though they serve the purpose of emotion elicitation materials, they were designed and developed as ordinary games. Along the process, several decisions were made concerning different aspects of each game, which inevitably affected the end results. Those decisions had an impact on the genre of each calibration game, as well as its graphical appearance and how complex the mechanics should be. A calibration game should induce a state of boredom at the beginning of the interaction, thus users should easily understand its mechanics in order to perceive the game as boring without a long exposure. It entails that the game mechanics must be easily understandable, preferably without much text or tutorials. Users should not spend a considerable amount of time learning the game, otherwise the concepts that induce boredom might be misunderstood the and desired emotional state would not be induced. Additionally all calibration games should not allow users to deliberately control the mechanics pace, since it was a key factor that was automatically controlled to induce stress towards the end of the session. Those constraints led the design of the calibration games towards more casual, 2D game mechanics. Even though games with similar characteristics exists, the proposed calibration games lack 3D content or a more complex interaction similar to those found in AAA COTS games, for instance. The genre/mechanics selected for the calibration games likely hinders several other genres and mechanics that could potentially be used as calibration games as well. The nature of the calibration games presented in this thesis do not cover the wide range of possible game types that exist, which limits its reach.

In that light, it could be argue that the calibration games proposed in this research only portrait emotions elicited from the specific selected genres/mechanics. Use of 2D, casual foundations for the calibration games could have conveyed a message of ``old games" to a segment of subjects/users, which is likely to impact their emotional reactions. On the other hand, the use of a more complex 3D game with sophisticated mechanics, e.g. Counter Strike, is likely to require a certain level of gaming skill from participants. In such case, it would impact the interactions of subjects that are not very familiar with gaming, which was the case of some participants in the heterogeneous groups presented in this research. As mentioned previously, individuals have different cultural views and expectations, so a more complex game makes it even harder to balance the design of a game with its intended purpose of inducing boredom and stress. The game Infinite Mario has been used in the validation process of the proposed method mainly because of its characteristic, e.g. easy to understand and play. Additionally it allowed more control over the content generation associated with its mechanics, so boring and stressful levels could be easily developed for the experiment mentioned in the thesis. It is plausible to believe that the emotion classification results obtained with Infinite Mario could be generalized to similar games, especially because Super Mario influenced a range of platformer games. However, as previously mentioned, the use of another 2D, casual game for the validation could limitation the generalization of results.

Another limitation of this research concerns the accuracy obtained by the proposed method in the classification of emotional states of boredom and stress. As presented in Chapters \ref{ch:experiment2} and \ref{ch:discussion}, the method achieved an accuracy of 61.6\%. Even though such classification rate has statistical significance that proves the method performs better than random guessing, such performance is still too low for commercial or even academic use. In its current state, the proposed method could not be used as a sole tool to detect the emotional state of users due to its noise. Additional measurements should accompany the proposed method to ensure a proper evaluation of the emotional context of subjects/users, e.g. questionnaires. The proposed method, however, could still be used as an insight mechanism to analyze large amounts of video footage in an automated way, for instance. Despite the best efforts invested in this research to design an accurate emotion detector, the complexity of the task and the amount of man-power available limited the exploration process. As opposed to aiming for a perfect tool, the research presented in this thesis focused on designing and rigorously evaluating each part of the proposed method. Such approach is expected to eventually guiding the construction of a more sophisticated emotion detector in the future.

Finally it is important to highlight technical limitations associated with the remote acquisition of physiological signals. The rPPG technique used in this research, as detailed in Section \ref{s:experiment1-study3}, is appropriate to deal with the natural behavior users show during the interaction with games. Such technique, however, was likely affected by other factors not scrutinized by this thesis. For instance, the 15 seconds long duration of each analysis window used for the estimation of HR may affect the results. The ideal length of the window (called window size) is not agreed upon in the literature \parencite{rouast2016remote}. In general, it depends on the characteristics of the rPPG technique being applied as well as the hardware configuration, such as camera framerate \parencite{roald2013estimation}. The statistical nature of ICA, part of the selected rPPG employed in this research, demands longer video samples to produce accurate results. The longer the video, however, the higher the chances of subject motion, which increases noise. A trade-off between the duration of the video segments and the estimation accuracy could be better investigated. Another factor is that the experimental setup used an external light source to minimize noise caused by changes in illumination, which should narrow the estimation error to causes as subject movement and/or facial activity. It is likely, however, that other factors might have impacted the estimation accuracy, such as facial hair, e.g. beard and hair over the forehead area, use of glasses, and skin color. Results obtained with this research were conceived in a laboratory-like environment with controlled light source, which limits the generalization of the conclusions. As detailed in Chapter \ref{ch:literature-rppg}, subject's movement and changes in illumination are significant challenges to the estimation accuracy of rPPG techniques. The use of a controlled light source, however, was deemed necessary to concentrate efforts on the remote detection of the emotional state, not on the noise caused by different illumination patterns.

%One potential limitation of our work is the internal validity. As previously described, the experiment was based on a one-group posttest design, which does not use a control group to measure the effects of the treatment. Such design could be criticized for having low internal validity, since it is not possible to unambiguously attribute cause and effect \parencite{kirk1982experimental}. A two-group approach could be suggested as having stronger internal validity, since it contains a control group and allows a less ambiguous conclusion. In the context of our research, however, any multiple group design implies the comparison of physiological signals and emotional perceptions among different people. Given the social and cultural background of the participants, it is virtually impossible to compare two groups of people regarding stress/boredom. People have different preferences, culture and expectations, which cause maturation and history threats to internal validity \parencite{trochim2001research}. Additionally the process of comparing variations of physiological signals among different subjects is a complex task, even when subjects are similar, e.g. same age and sex. As a consequence, a subject in a control group might present a set of variations of signals and classify a game as boring, while a similar subject in another group might classify the same game as not boring at all, presenting a different set of variations of signals. In that light, our experiment relies on a one-group experimental design to increase internal validity, since subjects were compared with themselves, which removes inter-subject differences.
