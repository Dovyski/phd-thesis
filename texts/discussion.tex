\chapter{Non-obtrusive detection of emotions}
\label{ch:discussion}

This chapter will "zoom out" from all the fine details of my thesis and give some perspective to the reader. I will come back to the introduction and present everything I did in a more abstract, high-level way.

\section{Game-based model for emotion detection}

This section will present the general idea of my approach and the steps I took to arrive at the final configuration. I will, on a high-level fashion, connect the results from each of the studies I conducted. I will also connect everything with my second experiment, showing how the approach was validaded.

\section{Insights into games as emotion elicitation}

Here I will present some of the insights I gained during the whole process. I can mention how HR actually changes in the calibration games, hinting that researchers could use this information to create better models, for instance. I will also mention how the use of remote estimations of HR is a good tool, however it is extremely affected by natural movement (as detailed in study 3).

\section{Enhancing questionnaires in game research}

Here I will connect my research with its practical use. I will try to show how it can be used as a tool to enhance the use of questionnaires in game research. I could say that this approach can be used to replace questionnaires, but our numbers do not allow such a bold statement at the moment.
