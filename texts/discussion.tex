\chapter{Non-obtrusive detection of emotions}
\label{ch:discussion}

The initial chapters of this thesis presented the theoretical foundations and the work needed to create a novel method for remote detection of emotions of users during the interaction with games. As highlighted by previous research, the understanding of human emotions, as well as the process of automatically detecting them, is the aim of several researchers in a many different fields. As detailed in Chapter \ref{ch:literature-games}, different theories have been proposed to model and study emotions in a variety of contexts, including those related to games. A considerable share of those theories is based on the human physiology, connecting emotional reactions to psychophysiological signals, e.g. heart rate. Several approaches have been proposed to put such models and theories into practice to achieve the ultimate goal of detecting what a person is feeling. Chapters \ref{ch:literature-face} and \ref{ch:literature-physiological}, for instance, describe the connection between emotions and their manifestations in the body, particularly the process of mapping measurable psychophysiologicalsignals into an emotional state.

Emotion detection is a complex and multidisciplinary problem that demands knowledge from many different areas. In this thesis, focus has been given to the field of games research. Results and contributions of this research are aimed at and discussed under the light of games research, however they are likely to be useful for scholars in other fields as well. This chapter presents and discusses the outcomes of a research focused on creating a non-obtrusive method for emotion detection, particularly in the context of games. The following sections also present the insights obtained during the systematic investigation and development of the proposed method, including a discussion on how they relate to games research and other areas.

\section{Game-based model for emotion detection}

% Contrary to the established notion that a group model is better, this research investigated the venue of a user-tailored approach, where individual characteristics might be preserved and better accounted for in a model.

This section will present the general idea of my approach and the steps I took to arrive at the final configuration. I will, on a high-level fashion, connect the results from each of the studies I conducted. I will also connect everything with my second experiment, showing how the approach was validaded.

\section{Insights outside games research}

Here I will present some of the insights I gained during the whole process. I can mention how HR actually changes in the calibration games, hinting that researchers could use this information to create better models, for instance. I will also mention how the use of remote estimations of HR is a good tool, however it is extremely affected by natural movement (as detailed in study 3).

\section{Enhancing questionnaires in game research}

Here I will connect my research with its practical use. I will try to show how it can be used as a tool to enhance the use of questionnaires in game research. I could say that this approach can be used to replace questionnaires, but our numbers do not allow such a bold statement at the moment.


%\begin{figure}[h]
%    \centering
%    \includegraphics[width=0.6\textwidth]{figures/model-inputs-set.png}
%    \caption{Overall structure of the user-tailored emotion detection model regarding input (user signals) and output (stress/boredom levels).}
%    \label{fig:model-inputs-set}
%\end{figure}

%The user-tailored model proposed for this research might have $N$ input signals, varying from physiological ones, e.g. HR, to non-physiological ones, e.g. facial actions and head movements. Figure \ref{fig:model-inputs-set} illustrates the overall structure of the model. In order to be used in the model, an input signal needs to be supported by previous work regarding emotion detection, as well as be validated within the process of the proposed game-based calibration phase. Time and scope constraints limit the amount of input signals that can be implemented, evaluated and used in this research. As a consequence, a study will be conducted to investigate, validate and initially implement two of those signals into the proposed model: HR and facial activity (which includes head movement, lips activity, etc).

%The techniques and works presented in chapter \ref{ch:literature-face}, which relate to face detection and emotion estimation, suggest that facial analysis is an important component of a multifactorial emotion detection model. Empirical analysis of the data from the first experiment also suggest that individualities regarding facial activities do exist and could be used to estimate emotional states on a user-tailored basis \parencite{bevilacqua2016variations}. As described in section \ref{ch:literature-face-emotion-detection}, facial actions, head movement, lips/eye/mouth activity and distance measurements of detected facial landmarks are viable and proven sources of information for emotion detection.

%Regarding physiological signals, results indicate that the average HR mean for players during the last minute of gameplay is greater than the average HR mean during the second minute of gameplay (chapter \ref{ch:experiment1}, section \ref{s:study3}). The findings are aligned with and reinforce previous research that indicates higher HR mean during stressful situations in a gaming context. The findings also suggest that changes in the HR during gaming sessions is a promising indicator of stress.

%The study will involve the definition of how those two signals will be used as inputs for the model. Facial actions, for instance, will probably be detected and measured by the euclidian distance of the facial landmarks. A vector containing the distances will be evaluated as the input for the model. Regarding the HR, its mean and standard deviation during a particular analysis window will be evaluated as input for the model. A software for the detection of those two signals will be created and used to analyse the video recordings of the first experiment (chapter \ref{ch:experiment1}). The inclusion or exclusion of a component of a signal, e.g. variations of the distances of the lips landmark points, will be based on the accuracy to detect them and the frequency they appear in boring and stressful part of the calibration games.
