\chapter{Games and emotions}

Research about games is a broad topic that involves different disciplines and definitions. In the context of this paper, a game is defined as a system in which players engage in an artificial conflict, defined by rules, that results in a goal \parencite{salen2004rules}. An artificial conflict is a set of challenges that the player must overcome towards the game goal, e.g. sort elements within a time constraint. In a game design perspective, the difficulty level of such challenges affects the emotional state of players, e.g. moments of boredom or anxiety/stress \parencite{schell2014art}. A challenge beyond the player's skill to address and overcome it causes anxiety, while the opposite results in disinterest, leading to boredom \parencite{chen2007flow}. An ideal challenge/skill balance produces an optimal experience and concentration state called flow \parencite{csikszentmihalyi1990psychology}. The flow theory is vastly mentioned in the literature, including its connection to engagement/immersion \parencite{brown2004grounded}, sense of presence \parencite{weibel2011immersion} and applicability in game design \parencite{cruz2017player, sweetser2005gameflow}. The evaluation of emotional states of player's in a game, including the level of enjoyment, requires methods that are able to capture information regarding emotions. The most commonly used techniques to obtain such data are self-reports (questionnaires) and physiological measurements \parencite{mekler2014systematic}. Questionnaires are subjective and they produce a shift in attention, which breaks the immersion of users and disturbs the measurements. Physiological signals, e.g. HR, on the other hand, are reliable sources of information and they can be measured without interruptions to the gaming experience \parencite{bousefsaf2013remote,yun2009game,rani2006empirical,tijs2008dynamic}.

%%%%%%%%%%%%%%%%%%%%%%%%%%%%%%%%%%%%%%%%%%%%%%%%%%%%%%%%%%%%%%%%%%%%%%%%%%%%%%%%%%%%%%%%%%%%%%%%%%%%%%%
\section{Stress, boredom and flow}
%%%%%%%%%%%%%%%%%%%%%%%%%%%%%%%%%%%%%%%%%%%%%%%%%%%%%%%%%%%%%%%%%%%%%%%%%%%%%%%%%%%%%%%%%%%%%%%%%%%%%%%

Computer games were proved to provoke alteration in the mean HR of players at stressful periods of gameplay \parencite{sharma2006assessment,rodriguez2015vr}


%%%%%%%%%%%%%%%%%%%%%%%%%%%%%%%%%%%%%%%%%%%%%%%%%%%%%%%%%%%%%%%%%%%%%%%%%%%%%%%%%%%%%%%%%%%%%%%%%%%%%%%
\section{Immersion, engagement and sense of presence}
%%%%%%%%%%%%%%%%%%%%%%%%%%%%%%%%%%%%%%%%%%%%%%%%%%%%%%%%%%%%%%%%%%%%%%%%%%%%%%%%%%%%%%%%%%%%%%%%%%%%%%%

%Presence, for instance, which describes the player's feeling of actually being in the game, is reported as an important aspect of engagement and immersion \parencite{weibel2011immersion}. Studies connected to training simulation \parencite{engstrom2016impact} also indicate that contextualization (increasing the sense of presence) might affect immersion positively.

The previously mentioned works try to relate physiological signals to emotional states. In games research, such initiatives are important to enhance the tooling available to better understand the player's experience while playing games. One concept that tries to model player experience is named flow, which was defined by Csikszentmihalyi \parencite{csikszentmihalyi1991flow}. Flow is an optimal experience achieved by the player during the interaction with a game. When in flow, a player will lose sense of time and space, focusing almost exclusively on the game being played, with a feeling of "being there".

The definition of flow, however, requires a more sophisticated interpretation, as investigated by further research. From a game design perspective, for instance, Schell \parencite{schell2014art} mentions that a progression of the player in the flow zone is desired, however it should not be a "straight line", but more of a cycle alternating among emotional states as stress, enjoyment and boredom. Different elements are connected to flow, such as engagement, immersion and sense of presence. Immersion, for instance, was defined by Brown and Cairns \parencite{brown2004grounded}, who used Grounded Theory to find it refers to the degree of involvement with a computer game. In that light, the authors theorized that a player will overcome barriers that limit his/her degree of involvement in immersion. After each barrier is broken, the sense of immersion deepens. The first barrier, for instance, is named engagement and it refers to the player willingness  to invest attention and energy to learn how to play the game. The concept of flow, as defined by Csikszentmihalyi, is an extreme state, which is only achieved when the player has overcame all previously mentioned barriers and is in a "total immersion" state. Such condition is rare to happen since it requires the highest level of attention from the player. As a consequence, engagement is more plausible and common during gaming experiences than flow.

The existence of several works \parencite{boyle2012engagement} related to understanding and defining what engagement/immersion is demonstrates the interest of researchers to broaden the view beyond flow alone. Presence, for instance, which describes the player's feeling of actually being in the game, is reported as an important aspect of engagement and immersion \parencite{weibel2011immersion}. Studies connected to training simulation \parencite{engstrom2016impact}, for instance, also indicate that contextualization (increasing the sense of presence) might affect immersion positively. Additionally to the intentions of understanding and defining engagement and immersion, studies also try to measure them. The majority of the approaches used for that are based on questionnaires, which are by nature subjectively answered by players. Additionally that approach usually breaks any sense of presence of the player since it requires a shift in attention away from the game, hence breaking or affecting the level of engagement/immersion as well. More quantitative approaches have been investigated to measure engagement, such as the use of physiological signals like HR \parencite{ravaja20051} and eye movement \parencite{jennett2008measuring}, for instance. The complexity in defining engagement and immersion is also reflected in the task of measuring it. Ravaja et al. \parencite{ravaja20051}, for instance, mentions the significant variation of physiological signals as an obstacle; signals increase during emotional arousal, but decrease in response to attention engagement, which makes the measurement of engagement a non-trivial process. It highlights the difficulties in correlating qualitative data to more abstract concepts as engagement, immersion and flow.

Research initiatives that help in the identification of the player emotional state without breaking the current level of engagement and immersion are desired. This research aims at using the link between the ANS and emotional regulation as the foundation to analyse physiological signals of a player, which will be remotely obtained and processed in a user-tailored fashion according to data previously collected in a game-based calibration phase.
