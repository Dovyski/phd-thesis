\chapter{Games and emotions}

Research about games is a broad topic that involves different disciplines and definitions. In the context of this paper, a game is defined as a system in which players engage in an artificial conflict, defined by rules, that results in a goal \parencite{salen2004rules}. An artificial conflict is a set of challenges that the player must overcome towards the game goal, e.g. sort elements within a time constraint. In a game design perspective, the difficulty level of such challenges affects the emotional state of players, e.g. moments of boredom or anxiety/stress \parencite{schell2014art}. A challenge beyond the player's skill to address and overcome it causes anxiety, while the opposite results in disinterest, leading to boredom \parencite{chen2007flow}. An ideal challenge/skill balance produces an optimal experience and concentration state called flow \cite{csikszentmihalyi1990psychology}. The flow theory is vastly mentioned in the literature, including its connection to engagement/immersion \cite{brown2004grounded}, sense of presence \cite{weibel2011immersion} and applicability in game design \cite{cruz2017player, sweetser2005gameflow}. The evaluation of emotional states of player's in a game, including the level of enjoyment, requires methods that are able to capture information regarding emotions. The most commonly used techniques to obtain such data are self-reports (questionnaires) and physiological measurements \cite{mekler2014systematic}. Questionnaires are subjective and they produce a shift in attention, which breaks the immersion of users and disturbs the measurements. Physiological signals, e.g. HR, on the other hand, are reliable sources of information and they can be measured without interruptions to the gaming experience \parencite{bousefsaf2013remote,yun2009game,rani2006empirical,tijs2008dynamic}.

%%%%%%%%%%%%%%%%%%%%%%%%%%%%%%%%%%%%%%%%%%%%%%%%%%%%%%%%%%%%%%%%%%%%%%%%%%%%%%%%%%%%%%%%%%%%%%%%%%%%%%%
\section{Stress, boredom and flow}
%%%%%%%%%%%%%%%%%%%%%%%%%%%%%%%%%%%%%%%%%%%%%%%%%%%%%%%%%%%%%%%%%%%%%%%%%%%%%%%%%%%%%%%%%%%%%%%%%%%%%%%

%%%%%%%%%%%%%%%%%%%%%%%%%%%%%%%%%%%%%%%%%%%%%%%%%%%%%%%%%%%%%%%%%%%%%%%%%%%%%%%%%%%%%%%%%%%%%%%%%%%%%%%
\section{Immersion, engagement and sense of presence}
%%%%%%%%%%%%%%%%%%%%%%%%%%%%%%%%%%%%%%%%%%%%%%%%%%%%%%%%%%%%%%%%%%%%%%%%%%%%%%%%%%%%%%%%%%%%%%%%%%%%%%%
