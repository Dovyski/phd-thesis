\chapter{Emotions and physiological signals}

%%%%%%%%%%%%%%%%%%%%%%%%%%%%%%%%%%%%%%%%%%%%%%%%%%%%%%%%%%%%%%%%%%%%%%%%%%%%%%%%%%%%%%%%%%%%%%%%%%%%%%%
\section{Introduction}
%%%%%%%%%%%%%%%%%%%%%%%%%%%%%%%%%%%%%%%%%%%%%%%%%%%%%%%%%%%%%%%%%%%%%%%%%%%%%%%%%%%%%%%%%%%%%%%%%%%%%%%

%%%%%%%%%%%%%%%%%%%%%%%%%%%%%%%%%%%%%%%%%%%%%%%%%%%%%%%%%%%%%%%%%%%%%%%%%%%%%%%%%%%%%%%%%%%%%%%%%%%%%%%
\section{Physiology of heart rate}
%%%%%%%%%%%%%%%%%%%%%%%%%%%%%%%%%%%%%%%%%%%%%%%%%%%%%%%%%%%%%%%%%%%%%%%%%%%%%%%%%%%%%%%%%%%%%%%%%%%%%%%

%%%%%%%%%%%%%%%%%%%%%%%%%%%%%%%%%%%%%%%%%%%%%%%%%%%%%%%%%%%%%%%%%%%%%%%%%%%%%%%%%%%%%%%%%%%%%%%%%%%%%%%
\section{Heart rate, stress and frustration}
%%%%%%%%%%%%%%%%%%%%%%%%%%%%%%%%%%%%%%%%%%%%%%%%%%%%%%%%%%%%%%%%%%%%%%%%%%%%%%%%%%%%%%%%%%%%%%%%%%%%%%%

The foundation of HR-based approaches for emotional estimation draws on the theory that physiological signals are linked to emotion regulation \parencite{appelhans2006heart,fenton2012emotion,schubert2009effects}. The automatic nervous system (ANS), a key system in the generation of physiological arousal \parencite{appelhans2006heart}, is subdivided into the sympathetic nervous system (SNS) and the parasympathetic nervous system (PNS). Those systems interact (often antagonistically) to produce variations in the physiological signals to prepare a person to react to a situation. The SNS is dominant during physical and psychological stress, triggering the body to alertness, increasing HR, for instance. This is commonly referred to as "fight or flight" response. The PNS, on the other hand, is dominant in periods of stability and relative safety, maintaining physiological signals at a lower degree of arousal, e.g. decreasing HR. The continuous changes between SNS and PNS impulses cause variations of HR and HRV \parencite{schubert2009effects}, which refers to beat-to-beat alternations in HR intervals.

Physiological signals, such as HR, are hard to fake because of their link with the ANS, differently from facial expressions \parencite{Landowska}, for instance. As a consequence, HR and its derivatives, such as HRV, have been used as reliable sources of information in different emotion estimation methods \parencite{kukolja2014comparative}. Additionally it has been used as an indication of perceived interest and confusion in mobile applications \parencite{xiao2015towards} and as a possible measurement of engagement by quantitative approaches \parencite{ravaja20051}. The significant variation of physiological signals, however, is an obstacle to its use in emotional estimation. Signals increase during emotional arousal, but decrease in response to attention engagement, which makes the measurement of engagement, for instance, a non-trivial process \parencite{ravaja20051}. Despite such challenges, the use of HR and HRV has been demonstrated in a continuous arousal monitor \parencite{grundlehner2009design} as well as a detection mechanism for mental and physical stress based on physical and mental tasks \parencite{vandeput2009heart,garde2002effects}. Results indicate higher HR during the mentally demanding task when compared to the rest period. The HR mean alone also has been shown to be a measurement of frustration in a game \parencite{rodriguez2015vr}. As previously mentioned, however, the use of sensors that require physical contact might disturb the user experience or be sensitive to noise. Even when users state they forgot about the use of physical sensors during a gaming session, researchers still face noisy data due to subject motion affecting the sensors \parencite{brogni2006variations}.
