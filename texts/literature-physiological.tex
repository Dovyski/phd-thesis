\chapter{Emotions and physiological signals}

%%%%%%%%%%%%%%%%%%%%%%%%%%%%%%%%%%%%%%%%%%%%%%%%%%%%%%%%%%%%%%%%%%%%%%%%%%%%%%%%%%%%%%%%%%%%%%%%%%%%%%%
\section{Introduction}
%%%%%%%%%%%%%%%%%%%%%%%%%%%%%%%%%%%%%%%%%%%%%%%%%%%%%%%%%%%%%%%%%%%%%%%%%%%%%%%%%%%%%%%%%%%%%%%%%%%%%%%

Heart rate (HR), for instance, is a source of information to measure emotional states \parencite{kivikangas2011review}, which can be used to detect emotions such as stress \parencite{choi2009using} or boredom \parencite{yamakoshi2007preliminary}.
%In games research, quantitative approaches already used HR to measure engagement \parencite{ravaja20051}, for instance.
%there are initiatives to measure such states and other elements, such as engagement/immersion \parencite{boyle2012engagement} and presence \parencite{weibel2011immersion}.

Computer games were proved to provoke alteration in the mean HR of players at stressful periods of gameplay \parencite{sharma2006assessment,rodriguez2015vr}


%%%%%%%%%%%%%%%%%%%%%%%%%%%%%%%%%%%%%%%%%%%%%%%%%%%%%%%%%%%%%%%%%%%%%%%%%%%%%%%%%%%%%%%%%%%%%%%%%%%%%%%
\section{Physiology of heart rate}
%%%%%%%%%%%%%%%%%%%%%%%%%%%%%%%%%%%%%%%%%%%%%%%%%%%%%%%%%%%%%%%%%%%%%%%%%%%%%%%%%%%%%%%%%%%%%%%%%%%%%%%

%%%%%%%%%%%%%%%%%%%%%%%%%%%%%%%%%%%%%%%%%%%%%%%%%%%%%%%%%%%%%%%%%%%%%%%%%%%%%%%%%%%%%%%%%%%%%%%%%%%%%%%
\section{Heart rate, stress and frustration}
%%%%%%%%%%%%%%%%%%%%%%%%%%%%%%%%%%%%%%%%%%%%%%%%%%%%%%%%%%%%%%%%%%%%%%%%%%%%%%%%%%%%%%%%%%%%%%%%%%%%%%%

The foundation of HR-based approaches for emotional estimation draws on the theory that physiological signals are linked to emotion regulation \parencite{appelhans2006heart,fenton2012emotion,schubert2009effects}. The automatic nervous system (ANS), a key system in the generation of physiological arousal \parencite{appelhans2006heart}, is subdivided into the sympathetic nervous system (SNS) and the parasympathetic nervous system (PNS). Those systems interact (often antagonistically) to produce variations in the physiological signals to prepare a person to react to a situation. The SNS is dominant during physical and psychological stress, triggering the body to alertness, increasing HR, for instance. This is commonly referred to as "fight or flight" response. The PNS, on the other hand, is dominant in periods of stability and relative safety, maintaining physiological signals at a lower degree of arousal, e.g. decreasing HR. The continuous changes between SNS and PNS impulses cause variations of HR and HRV \parencite{schubert2009effects}, which refers to beat-to-beat alternations in HR intervals.

Physiological signals, such as HR, are hard to fake because of their link with the ANS, differently from facial expressions \parencite{Landowska}, for instance. As a consequence, HR and its derivatives, such as HRV, have been used as reliable sources of information in different emotion estimation methods \parencite{kukolja2014comparative}. Additionally it has been used as an indication of perceived interest and confusion in mobile applications \parencite{xiao2015towards} and as a possible measurement of engagement by quantitative approaches \parencite{ravaja20051}. The significant variation of physiological signals, however, is an obstacle to its use in emotional estimation. Signals increase during emotional arousal, but decrease in response to attention engagement, which makes the measurement of engagement, for instance, a non-trivial process \parencite{ravaja20051}. Despite such challenges, the use of HR and HRV has been demonstrated in a continuous arousal monitor \parencite{grundlehner2009design} as well as a detection mechanism for mental and physical stress based on physical and mental tasks \parencite{vandeput2009heart,garde2002effects}. Results indicate higher HR during the mentally demanding task when compared to the rest period. The HR mean alone also has been shown to be a measurement of frustration in a game \parencite{rodriguez2015vr}. As previously mentioned, however, the use of sensors that require physical contact might disturb the user experience or be sensitive to noise. Even when users state they forgot about the use of physical sensors during a gaming session, researchers still face noisy data due to subject motion affecting the sensors \parencite{brogni2006variations}.

%%%%%%%

The relation between HR and emotional states, such as stress and boredom, is a recurrent research topic. HR has been used in different forms, for instance as player input for games \parencite{stockhausen2013beats}, as an indication of perceived interest and confusion in mobile applications \parencite{xiao2015towards}, as triangulation of phychophysiological emotional reactions to digital media stimuli \cite{nogueira2015annotation} and as possible measurement of engagement by quantitative approaches \parencite{ravaja20051}. The foundation of such HR-based approaches for emotional estimation lays on the theory that physiological signals are linked to emotion regulation \parencite{fenton2012emotion,schubert2009effects}. The automatic nervous system (ANS), a key system in the generation of physiological arousal \parencite{appelhans2006heart}, is subdivided into the sympathetic nervous system (SNS) and the parasympathetic nervous system (PNS). Those systems interact (often antagonistically) to produce variations in the physiological signals to prepare a person to react to a situation. The SNS is dominant during physical and psychological stress, triggering the body to alertness, increasing HR, for instance. This is commonly referred to as ``fight or flight" response. The PNS, on the other hand, is dominant in periods of stability and relative safety, maintaining physiological signals at a lower degree of arousal, e.g. decreasing HR.

The continuous changes between SNS and PNS impulses cause variations of HR and heart rate variability (HRV) \parencite{schubert2009effects}, which refers to beat-to-beat alternations in HR intervals. \textcite{vandeput2009heart} demonstrate the use of HR and HRV to detect mental and physical stress. Three demanding activities (a postural task, a mental task and a task that is a combination of both) are used and for almost all HR measures obtained, the demanding activities can be distinguished from the rest period. The authors also point out that mental stress decreased high frequency components of the HRV interval (i.e. $HRV_{HF}$), while increasing low frequency ones (i.e. $HRV_{LF}$). A similar experiment was conducted by \textcite{garde2002effects} involving two tasks: one being mental and physically demanding (digital version of the Stroop color word test \parencite{golden1978stroop}) and other being only physically demanding. The authors confirm the findings of \textcite{vandeput2009heart} by showing higher HR, increased $HRV_{LF}$ and decreased $HRV_{HF}$ during the mentally demanding task when compared to the rest period.

\textcite{bousefsaf2013remote} also use a digital version of the Stroop test in a similar experiment. A combination of HR, HRV and HRV\textsubscript{HF} is used to estimate the stress state of subjects. The resulting stress state curve tends to decrease during the rest period and increase during stress sessions (as did the HR), in accordance with the previously mentioned works; additionally the self-reported answers to questionnaires present significant differences of stress level in the rest and in the stress sessions as well. \textcite{mcduff2014remote,mcduffcogcam} also use HR and its variants in order to measure cognitive stress during computer tasks. According to the authors, the average heart rate and breathing rate are not significantly different in any case, which differs from the findings of previously mentioned work. The variations of HRV\textsubscript{LF} and HRV\textsubscript{HF}, however, are significantly different during the cognitive tasks compared to the rest period; higher HRV\textsubscript{LF} and lower HRV\textsubscript{HF} power are found in both cognitive tasks compared to the rest period, which aligns with findings of previous work. Authors also point out that the stress predictions made by their model are consistent with the self-reported answers.

\textcite{grundlehner2009design} present a real-time, continuous arousal monitor. Using a wireless sensor network for signal acquisition, the authors record and use four signals from subjects to estimate arousal: electrocardiogram (ECG), respiration, skin conductance and skin temperature. The ECG is used to calculate HRV, which is then used in the estimation. A regression analysis is performed to identify the importance of the features in the estimation of arousal. HRV\textsubscript{LF} and HRV\textsubscript{HF} are not significant when compared to the other signals (e.g. skin conductance), while the standard deviation of HRV presents a significant weight. The arousal prediction matches the hypothesized arousal events marked by the authors in each of the experiment parts (e.g. start of noise in the audio part).

Finally \textcite{rodriguez2015vr} demonstrate that HR mean alone can be used as a measurement of frustration in a game. The authors use a game to induce frustration on subjects, which cause changes in their HR mean. The changes are a direct consequence of the game, as proved by the authors.
